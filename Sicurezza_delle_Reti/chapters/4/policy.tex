\chapter{Politiche di sicurezza}

La \textit{gestione della sicurezza} è un \textit{processo formale} 
per rispondere alle domande:
\begin{itemize}
    \item quali sono i beni da proteggere
    \item quali sono le possibili minacce
    \item come si possono contrastare le minacce 
\end{itemize}

\noindent Questo processo ha natura iterativa, ed è contenuto nella ISO 31000; in 
questa norma viene descritto un \textit{modello per la gestione della sicurezza 
delle informazioni} che comprende le seguenti fasi:
\begin{itemize}
    \item \textbf{Plan:}
    \begin{itemize}
        \item stabilire politiche, processi e procedure di sicurezza
        \item eseguire la valutazione del rischio 
        \item sviluppare un piano di trattamento del rischio 
    \end{itemize}
    \item \textbf{Do:}
    \begin{itemize}
        \item implementare il piano di trattamento del rischio
    \end{itemize}
    \item \textbf{Check:}
    \begin{itemize}
        \item monitorare e mantenere il piano di trattamento del rischio
    \end{itemize}
    \item \textbf{Act:}
    \begin{itemize}
        \item mantenere e migliorare la gestione dei rischi 
        \item risposta ad incidenti, analisi di vulnerabilità e riprendere il ciclo iterativamente
    \end{itemize}
\end{itemize}

\newpage
\section{Definizioni}
\begin{itemize}
    \item \textbf{Rischio:} esprime la possibilità che un attacco causi danni ad una 
    organizzazione
    \item \textbf{Risorsa:} tutto ciò che necessita di essere protetto
    \begin{itemize}
        \item \textit{hw}
        \item \textit{sw}
        \item \textit{reputazione}
    \end{itemize}
    
    \noindent La valutazione di una risorsa viene fatta in base:
    \begin{itemize}
        \item ai costi da sostenere per sostituire la risorsa nel caso non sia più disponibile 
        \item perdita di incassi in caso di attacco 
    \end{itemize}
    \item \textbf{Vulnerabilità:} punti deboli che possono essere sfruttati per causare danni al 
    sistema; possono essere classificate come:
    \begin{itemize}
        \item critico 
        \item moderato 
        \item basso
    \end{itemize}
\end{itemize}











