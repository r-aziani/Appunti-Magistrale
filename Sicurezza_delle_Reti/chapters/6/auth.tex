\chapter{Autenticazione}
Processo per determinare se un utente, un'applicazione o un processo che agisce per conto di un utente 
\textbf{è effettivamente chi o cosa dichiara di essere}; si fa un confronto tra le credenziali 
inserite e quelle memorizzate nel sistema.

\section{Principi di autenticazione}
\begin{itemize}
    \item \textbf{Identità digitale:} è la rappresentazione unica di un soggetto, consiste in 
    un attributo o un insieme di attributi che descrivono univocamente un soggetto 
    \item \textbf{Prova dell'identità:} stabilisce che un soggetto è chi afferma di essere ad un determinato 
    livello di certezza
    \item \textbf{Autenticazione digitale:} il processo di determinazione della validità di uno o più 
    autenticatori utilizzati per rivendicare un'identità digitale 
\end{itemize}

\section{Requisiti di sicurezza}
\begin{itemize}
    \item \textbf{Di base:} 
    \begin{itemize}
        \item \textbf{identificare} gli utenti del sistema informativo e i processi che agiscono per loro conto 
        \item autenticare (o \textbf{verificare}) le identità di tali utenti 
    \end{itemize} 
    \item \textbf{Derivati:}
    \begin{itemize}
        \item utilizzare l'\textbf{autenticazione a più fattori}
        \item impiegare meccanismi resistenti ad \textbf{attacchi di replay}
        \item impedire il riutilizzo degli \textit{id} per un certo intervallo di tempo 
        \item disabilitare gli \textit{id} dopo un periodo di inattività
        \item applicare una \textbf{complessità minima} della password
        \item proibire il \textbf{riutilizzo} delle password 
        \item memorizzare e trasmettere solo password \textbf{criptate}
        \item \textbf{oscurare il feedback} delle informazioni di autenticazione
    \end{itemize}
\end{itemize}

\section{Metodi di autenticazione}
\begin{itemize}
    \item \textbf{qualcosa che uno sa:} password, \dots
    \item \textbf{qualcosa che uno ha:} token $\rightarrow$ smartcard, chiave fisica, \dots
    \item \textbf{qualcosa che l'individuo è:} tratto biometrico
    \item \textbf{qualcosa che l'individuo fa:} biometria dinamica
\end{itemize}

\section{Password}
Lo schema standard \textit{id-password} è soggetto a diverse vulnerabilità:
\begin{itemize}
    \item canale tra client e server 
    \item compromissione del client 
    \item compromissione del server 
    \item social engeneering 
    \item password deboli 
\end{itemize}

\noindent Esistono diversi tipi di attacco alle password:
\begin{itemize}
    \item attacchi a \textbf{dizionario} 
    \item \textbf{forza bruta}
    \item uso della \textbf{stessa password} per più servizi
\end{itemize}

\subsection{Memorizzazione delle password}
Nello vecchio sistema UNIX veniva memorizzato nel file \texttt{/etc/passwd} l'hash delle password 
insieme al suo id.

\noindent Oggi, la password viene salvata come \textbf{hash} nel file \textbf{\texttt{/etc/shadow}},
che è \textbf{leggibile solo da \textit{root}}.

\noindent Per proteggersi da attacchi di forza bruta, è possibile usare un \textbf{salt}: invece che memorizzare 
l'hash della password, ad esso viene concatenato all'inizio e/o alla fine un \textit{numero casuale} di 
lunghezza arbitraria; vengono usati salt fino a 48 bit.

\subsection{Cracking delle password}
\begin{itemize}
    \item attacchi di dizionario 
    \item tabelle \textit{rainbow}: contengono hash precalcolati di diversi dizionari, con già inseriti 
    dei salt 
\end{itemize}

\noindent La maggioranza degli attacchi \textbf{sfrutta password deboli}; la soluzione è costringere 
gli utenti ad usare \textbf{password complesse}.

\subsection{Utilizzo delle password su canali non sicuri}
Esistono tre approcci:
\begin{itemize}
    \item \textbf{One-time-password}
    \item \textbf{Challenge response:} chi si autentica dimostra di conoscere un segreto (challenge); 
    di norma vengono usati anche timestamp per evitare attacchi di replay
    \item \textbf{Zero knowledge proof:} chi si autentica fornisce una prova di conoscere la password 
    senza fornirla esplicitamente 
\end{itemize}

\section{Altri metodi di autenticazione}
\begin{itemize}
    \item \textbf{Smart token:} dispositivi dotati di \textbf{capacità computazionale}, possono fare diversi tipi di controlli 
    \item \textbf{Smart card:} forniscono informazioni una volta inserite nel sensore, non hanno sistemi di calcolo 
    \item \textbf{Autenticazione biometrica}
    \item \textbf{Autenticazione reciproca:} permettono ad entrambe le parti di accertarsi reciprocamente della propria identità
\end{itemize}


\section{Problemi di sicurezza dell'autenticazione}
\begin{itemize}
    \item \textbf{Intercettazione}, ad esempio con un attacco che implica la vicinanza fisica di utente ed attaccante
    \item \textbf{Attacchi host:} diretti al file dell'utente in cui sono contenute le password/token/\dots
    \item \textbf{Replay:} l'avversario ripete una risposta dell'utente acquisita in precedenza, senza che il server 
    sia in grado di distinguerla; si può contrastare con tre approcci:
    \begin{itemize}
        \item allegare un \textbf{numero di sequenza} a ciascun messaggio
        \item \textbf{timestamp} per verificare la validità
        \item \textbf{challenge-response:} quando una parte invia un messaggio, deve contenere il
        \textit{nonce} del messaggio precedente
    \end{itemize}
    \item \textbf{Attacchi client:} l'avversario cerca di autenticarsi senza accedere all'host remoto 
    \item \textbf{DoS:} tenta di disabilitare un servizio 
\end{itemize}






