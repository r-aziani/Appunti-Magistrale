\chapter{IDS (Intrusion Detection System)}

Dato che i firewall non possono proteggere da tutte le minacce e un \textit{instrusione}
non è un evento impossibile, è fondamentale e rilevarla e porre rimedio; viene 
fatto tramite gli \textbf{Intrusione Detection Systems}:
\begin{itemize}
    \item \textbf{sistema per identificare individui/servizi che usano un computer o una 
    rete senza autorizzazione}; il controllo è esteso anche ad utenti autorizzati, ma 
    che \textbf{non rispettano i loro privilegi}
    \item viene fatto monitorando il traffico di rete; può collaborare con il firewall
    \item è utile anche per fornire informazioni utili su intrusioni avvenute, fare diagnosi e correggere debolezze
\end{itemize} 

\noindent È importante ricordare che gl'IDS \textbf{non è un sistema di protezione 
ma di rilevazione} delle intrusioni o di attacchi provenienti dall'esterno.

\subsubsection{Caratteristiche funzionali}
\begin{itemize}
    \item \textbf{IDS passivi:}
    \begin{itemize}
        \item checksum crittografici 
        \item riconoscimento di pattern 
    \end{itemize}
    \item \textbf{IDS attivi:}
    \begin{itemize}
        \item \textit{learning} $\rightarrow$ analisi statistica del sistema 
        \item \textit{monitoring} $\rightarrow$ analisi attiva del traffico di dati, azionim \dots
        \item \textit{reaction} $\rightarrow$ confronto con parametri statistici (reazione scatta al superamento di una soglia)
    \end{itemize}
\end{itemize}

\section{Caratteristiche topologiche}

\begin{itemize}
    \item \textbf{HIDS (host-based IDS):}
    \begin{itemize}
        \item analisi dei log 
        \item attivazione di strumenti di monitoraggio interni al S.O.
    \end{itemize}
    \item \textbf{NIDS (network-based IDS):}
    \begin{itemize}
        \item attivazione di strumenti di monitoraggio del traffico di rete 
    \end{itemize}
\end{itemize}

\subsubsection{Componenti di un NIDS}
\begin{itemize}
    \item \textbf{Sensor:} controlla il traffico e log per individuare pattern sospetti; attiva i 
    security event quando necessario
    \item \textbf{Director:} coordina i sensor 
    \item \textbf{IDS message system:} consente la comunicazione sicura tra i componenti dell'IDS
\end{itemize}



\section{Valutare un IDS}
Per valutare l'efficienza di un IDS occorre conoscere due parametri:
\begin{itemize}
    \item \textbf{Accuratezza:} $allarmi Corretti / allarmi Totali$
    \item \textbf{Completezza:} $allarmi Corretti / intrusioni Totali$
\end{itemize}

\noindent Occore sempre bilanciare i \textit{falsi negativi}
e i \textit{falsi positivi}, dato che questi parametri sono correlati inversamente.












