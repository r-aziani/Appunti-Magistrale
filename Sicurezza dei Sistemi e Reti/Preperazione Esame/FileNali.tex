\documentclass{report}
\usepackage{graphicx} % Required for inserting images
\usepackage[italian]{babel}
\usepackage{tikz}
\usepackage{hyperref}
\usepackage{amsmath}
\usepackage{xcolor}
\usepackage{float}
\usepackage{soul}
\usepackage{listings} % Per evidenziare il codice

\definecolor{lightgray}{rgb}{0.9,0.9,0.9} % Definizione colore sfondo
\definecolor{darkgreen}{rgb}{0.0, 0.5, 0.0}

\lstset{
    backgroundcolor=\color{lightgray}, % Sfondo grigio
    basicstyle=\ttfamily, % Font monospaziato
    % frame=single, % Bordo attorno al codice
    tabsize=4, % Dimensione tabulazione
    breaklines=true, % Permette di andare a capo automaticamente
    numbers = left,
    numberstyle=\small\color{gray}
}

\title{\huge\textbf{{Sicurezza dei Sistemi e delle Reti}}}
\date{File di Nali}

\begin{document}

\maketitle

\tableofcontents
\newpage

\chapter{Politiche di Sicurezza}
\section{Definizione Politica di sicurezza}

La \textit{gestione della sicurezza} è un \textit{processo formale} 
per rispondere alle domande:
\begin{itemize}
    \item quali sono i beni da proteggere
    \item quali sono le possibili minacce
    \item come si possono contrastare le minacce 
\end{itemize}

\noindent Questo processo ha natura iterativa, ed è contenuto nella ISO 31000; in 
questa norma viene descritto un \textit{modello per la gestione della sicurezza 
delle informazioni} che comprende le seguenti fasi:
\begin{itemize}
    \item \textbf{Plan:}
    \begin{itemize}
        \item stabilire politiche, processi e procedure di sicurezza
        \item eseguire la valutazione del rischio 
        \item sviluppare un piano di trattamento del rischio 
    \end{itemize}
    \item \textbf{Do:}
    \begin{itemize}
        \item implementare il piano di trattamento del rischio
    \end{itemize}
    \item \textbf{Check:}
    \begin{itemize}
        \item monitorare e mantenere il piano di trattamento del rischio
    \end{itemize}
    \item \textbf{Act:}
    \begin{itemize}
        \item mantenere e migliorare la gestione dei rischi 
        \item risposta ad incidenti, analisi di vulnerabilità e riprendere il ciclo iterativamente
    \end{itemize}
\end{itemize}

\subsection{Definizioni}
\begin{itemize}
    \item \textbf{Rischio:} esprime la possibilità che un attacco causi danni ad una 
    organizzazione
    \item \textbf{Risorsa:} tutto ciò che necessita di essere protetto
    \begin{itemize}
        \item \textit{hw}
        \item \textit{sw}
        \item \textit{reputazione}
    \end{itemize}
    
    \noindent La valutazione di una risorsa viene fatta in base:
    \begin{itemize}
        \item ai costi da sostenere per sostituire la risorsa nel caso non sia più disponibile 
        \item perdita di incassi in caso di attacco 
    \end{itemize}
    \item \textbf{Vulnerabilità:} punti deboli che possono essere sfruttati per causare danni al 
    sistema; possono essere classificate come:
    \begin{itemize}
        \item critico 
        \item moderato 
        \item basso
    \end{itemize}
\end{itemize}



\section{Domanda MAC, DAC e RBAC}
\begin{center}
    \textit{definire l'utilizzo delle politiche di sicurezza basate su MAC, DAC e RBAC}
\end{center}
\noindent Gli approcci DAC,MAC e RBAC fanno parte delle politiche di controllo degli accessi e forniscono approcci diversi
a seconda del contesto di applicazione
\subsection{DAC (Discretionary Access Control):}
\noindent Il controllo dell'accesso viene fatto sull' \textbf{identità del soggetto richiedente} e delle \textbf{regole di accesso}.
Definito \textit{discrezionale} poichè un'entità potrebbe avere i privilegi di accessi che le permettono, a sua volta, di concedere l'accesso ad un'altra entità.

\noindent Si può rappresentare mediante una matrice, dove:
\begin{itemize}
    \item le colonne rappresentano i soggetti
    \item le righe gli oggetti
    \item ogni cella specifica i diritti di accesso di quel soggetto a quel determinato oggetto
\end{itemize}


\subsection{MAC (Mandatory Access Control):} 
La politica di controllo degli accesssi MAC, o Mandatory Access Controll si basa sul confronto tra \textbf{etichette di sicurezza} che indicano quanto sono sensibili le risorse
e \textbf{autorizzazioni di sicurezza} che indicano quali entità del sistema sono idonee ad accedere a quali risorse.

\noindent Questa politica è definita \textit{mandatoria} poichè un'entità che possiede l'accesso a una risorsa non può estendere il permesso di acesso a un'altra entità,
può farlo solo l'amministratore di sistema.

\noindent I sistemi MAC si dividono in:
\begin{itemize}
    \item \textbf{Multilevel security systems:} consiste in una struttura verticale di sistemi di sicurezza, agli utent viene assegnato un livello e possono accedere solo a risorse con un livello uguale o inferiore
    \item \textbf{Multilateral security systems:} l'accesso viene assegnato in base a segmenti che formano gruppi costituiti da livelli di sicurezza e parole in codice
    \begin{itemize}
        \item si ottiene una struttura orizzontale, che contiene livelli di sicurezza verticali aggiuntivi
    \end{itemize}
\end{itemize}

\noindent Vantaggi:
\begin{itemize}
    \item molto sicuro, a prova di manomissione
    \item gli utenti non possono fare modifiche
    \item controllo automatizzato
    \item i dati non possono essere modificati senza apposita autorizzazione
\end{itemize}

\noindent Svantaggi:
\begin{itemize}
    \item richiede una pianificazione dettagliata e un lavoro amministrativo
    \item controllo e aggiornamento dei dati di accesso
    \item manutenzione per aggiunta di nuovi dati o utenti e relative modifiche (elevato carico di lavoro per l'amministratore)
\end{itemize}

\subsection{RBAC (Role Based Access Control):}
\noindent Introduce il concetto di \textbf{ruolo}, ovvero una funzione che può essere assiciata a uno o più utenti (gli utenti possono avere più ruoli)
e una \textbf{sessione}, ovvero una mappatura tra utente e un sottoinsieme di ruoli a cui è assegnato.

\noindent I ruoli di un utente possono fornire o meno accesso a determinate risorse.

\noindent Quattro tipi di entità:
\begin{itemize}
    \item \textbf{Utente:} una persona che ha accesso al sistema, ogni individuo ha un ID associato
    \item \textbf{Ruolo:} funzione lavorativa all'interno dell'organizzazione
    \item \textbf{Autorizzazione:} approvazione di una modalità di accesso ad uno o più oggetti
    \item \textbf{Sessione:} mappatura tra utente e un sottoinsieme dei ruoli a cui è assegnato
\end{itemize}

\section{Fare cenni sull'utilizzo di tali politiche nei Sistemi Operativi moderni}
\subsection{Unix security model}
In Linux ci sono tre entità da considerare:
\begin{itemize}
    \item \textbf{Soggetto:} può essere un utente o un processo 
    \item \textbf{Oggetto:} file, cartelle, \dots
    \item \textbf{Operazioni consentite:} lettura, scrittura, esecuzione 
\end{itemize}

\noindent In Unix, ogni utente ha associato un id univoco, detto \textbf{UID}; può appartenere a 
gruppi di utenti, anch'essi identificati da un id univoco detto \textbf{GID}. Tutti gli utenti 
appartenenti ad un gruppo possono condividere tra loro oggetti.

\noindent Ad ogni file è assegnato un unico utente proprietario e un unico gruppo proprietario. L'autorizzazione 
viene concessa mediante una ACL che identifica le operazioni che i soggetti possono fare.

\subsubsection{Processi in Linux}
Ogni processo è isolato dagli altri e non possono accedere alla memoria altrui. Ogni processo viene 
eseguito con le autorizzazione dell'UID dell'utente che lo sta eseguendo. 

\noindent Nel momento della creazione, ad ogni processo sono assegnati tre ID (inizialmente tutti uguali 
all'UID):
\begin{itemize}
    \item \textbf{Effective UID:} determina le autorizzazioni per il processo 
    \item \textbf{Real UID:} determina l'utente che ha avviato il processo 
    \item \textbf{Saved UID:} EUID prima di eventuali modifiche
\end{itemize}

\noindent L'utente \textit{root} può cambiare EUID/RUID/SUID a valori arbitrari; utenti non privilegiati
possono cambiare EUID solo a RUID o SUID

\subsubsection{Unix file access control}
Le modifiche agli ID sono apportate mediante i comandi \textit{setUID} e \textit{setGID}; questa modifica 
permette ai programmi non privilegiati di accedere a risorse generalmente non accessibili.

\noindent Le directory possono aver impostato uno \textit{\textbf{sticky bit}}: specifica che solo il proprietario di un file nella cartella
può apportare una modifica a quel file 

\noindent Il \textit{\textbf{superuser}} è esente dalle consuete restrizioni di controllo degli accessi, ha 
accesso a tutto il sistema.


\subsection{Windows security architecture}
L'architettura di sicurezza di Windows è basata su più entità:
\begin{itemize}
    \item \textbf{\textit{Security Reference Model (SRM):}} componente che in modalità kernel esegue 
    controlli delle autorizzazioni e manipola i privilegi degli utenti
    \item \textbf{\textit{Local Security Authority (LSA):}} risiede in un processo utente, è responsabile 
    dell'applicazione della politica di sicurezza locale, tra cui:
    \begin{itemize}
        \item criteri per le password, come complessità e tempi di scadenza 
        \item politica di controllo $\rightarrow$ specifica quali operazioni su quali oggetti vadano controllate 
        \item impostazioni dei privilegi
    \end{itemize}
    \item \textbf{\textit{Security Account Manager (SAM):}} è un database che archivia i dati degli account e le 
    informazioni di sicurezza su entità locali e gruppi
    \item \textbf{\textit{Active Directory (AD):}} implementa il protocollo LDAP (\textit{Lightweight Directory Access Protocol})
\end{itemize}

\subsubsection{Windows security model}
Windows ha un complesso sistema di controllo dell'accesso; ogni oggetto ha ACL per permettere autorizzazioni 
granulari ad utenti e/o gruppi di utenti.

\subsubsection{Security descriptor}
Ogni oggetto ha un \textit{security descriptor} che contiene:
\begin{itemize}
    \item \textbf{\textit{security identifier} (SID)} per il possessore e il gruppo primario dell'oggetto (SID è 
    associato univocamente ad ogni utente) 
    \item \textbf{\textit{discretionary ACL} (DACL):} diritti di accesso per gli utenti e i gruppi
    \item \textbf{\textit{system ACL} (SACL):} tipi di accesso che generano log
\end{itemize}

\noindent Ad ogni processo viene inoltre associato un insieme di \textbf{token}, che prende il nome di 
\textbf{\textit{security context}}. Nel momento in cui un processo vuole accedere ad un oggetto, presenta 
il suo insieme di token e il sistema controlla se il security context abbia o meno accesso a tale risorsa 
in base al \textit{security descriptor} dell'oggetto.



\chapter{Set-UID}






\chapter{Malware}
\noindent Una definizione informale per malware potrebbe essere quella di \textit{programma malevolo},
solitamente inserito di nascosto in un sistema, che ha lo scopo di compromettere la \textbf{riservatezza}, l'\textbf{integrità} o la \textbf{disponibilità} 
del sistema stesso.

\noindent I malware sono classificati in base a:
\begin{itemize}
    \item \textbf{Propagazione:} software, rete, social engeneering
    \item \textbf{Azioni sui dati colpiti:} corruzione, furto, crittografia
    \item \textbf{Attack kit:} strumenti già pronti per attaccare
    \item \textbf{Attori e/o motivazioni dell'attacco}
\end{itemize}

\section{Differenza tra due tipologie di malware (virus e worm) + esempio}
\noindent Sono stati chiesti questi due ma per sicurezza aggiungo anche gli altri
\subsection{Trojan}
È un programma che ha un effetto evidente e atteso dall'utente, che ha però anche un 
effetto \textbf{nascosto} che viola le politiche di sicurezza e che viene condotto senza l'autorizzazione 
dell'utente

\subsection{Virus}
È un codice che può replicarsi modificando altri file o programmi per inserire codice in 
grado di replicarsi a sua volta; questa \textbf{proprietà di replicazione} è ciò distingue 
i virus dagli altri tipi di malware. Non svolge nessuna azione evidente, ma cerca di rimanere 
nell'ombra.

\noindent La replica richiede un certo tipo di assistenza da parte dell'utente, come ad esempio
cliccare su un allegato.

\noindent Un virus è composto da tre parti:
\begin{itemize}
    \item \textbf{Meccanismo di infezione}
    \item \textbf{Trigger:} evento che determina quando il payload viene attivato
    \item \textbf{Payload:} cosa fa il virus (oltre a diffondersi)
\end{itemize}

\noindent I virus attraversano quattro fasi:
\begin{enumerate}
    \item \textbf{Dormiente:} il virus è inattivo in attesa di essere attivato 
    \item \textbf{Scatenante:} il virus viene attivato 
    \item \textbf{Propagazione:} il virus inserisce una copia di sé stesso in certe parti del sistema; ogni programma 
    infetto conterrà ora un altro virus che entrerà a sua volta in fase di propagazione 
    \item \textbf{Esecutiva:} la funzione viene eseguita
\end{enumerate}

\subsubsection{Vettori di infezione}
I principali vettori di infezione sono:
\begin{itemize}
    \item \textbf{Boot sector} di dispositivi esterni; il codice è inserito nel boot sector e viene 
    eseguito in fase di avvio 
    \item \textbf{Eseguibili}
    \item \textbf{File macro:} il virus si attacca ai documenti per propagarsi 
\end{itemize}

\subsubsection{Esempio noto in letteratura:}


\subsection{Worm}
I worm sono programmi \textit{stand alone} (a differenza dei virus che devono essere attivati 
da un qualche evento) in grado di replicarsi.

\noindent Le fasi di esecuzione sono:
\begin{itemize}
    \item \textbf{Probing:} cerca informazioni sulla macchina 
    \item \textbf{Expoloitation:} sfrutta le informazioni raccolte per trovare vulnerabilità
    \item \textbf{Replicazione}
    \item \textbf{Attacco} (payload)
\end{itemize}

\subsubsection{Esempio noto in litteratura:}



\subsection{Drive-by-download}
Sfruttano \textbf{vulnerabilità del browser} per installare codice malevolo ad insaputa dell'utente 
nel momento in cui visita la pagina web dell'attaccante. 

\subsection{Clickjacking}
L'attaccante intercetta un \textit{click} dell'utente per costringerlo a fare delle cose 
contro la sua volontà.

\subsection{Zombie e botnet}
Lo \textit{zombie} è una singola macchina, mentre la \textit{botnet} è un insieme di macchine 
zombie controllate da una singola entità; vengono usate per fare DDoS, phishing, spamming, \dots

\subsection{Rootkit}
È un insieme di programmi installati su un sistema per mantenere l'accesso ad un sistema, ad 
esempio, con privilegi di amministratore, nascondendo le prove della sua presenza e aggirando 
i meccanismi di controllo.

\noindent Permettono di fare attacchi anche con scarse conoscenze tecniche.

\subsection{Scareware}
Software che hanno lo scopo di diffondere shock, ansia e/o la percezione di una minaccia; sono 
un attacco di \textit{social engeneering}.

\subsection{Ransomware}
Software che tiene in ostaggio il sistema per richiedere un riscatto all'utente, spesso tramite
cifratura.

\subsection{Vulnerabilità zero-day}
Si intende un'exploit non ancora nota e che non ha quindi una contromisura.

\subsection{Spear phishing}
Viene \textbf{studiato nel dettaglio il bersaglio}, in modo tale da fare del phishing più mirato 
ed efficace.

\subsection{Spyware}
Malware che raccoglie piccole informazioni alla volta sugli utenti a loro insaputa,
come ad esempio un \textit{keylogger}.

\subsection{APT - Advanced Persistent Threats}
\begin{itemize}
    \item \textit{\textbf{Advanced:}} è un'applicazione con un ampia varietà di tecnologie di intrusione e malware
    \item \textit{\textbf{Persistent:}}
    attacchi per un periodo di tempo prolungato verso il target
    \item \textit{\textbf{Target:}} target selezionati in modo accurato
\end{itemize}

\noindent Le fasi principali di un attacco tramite APT sono:
\begin{itemize}
    \item \textbf{Ricognizione:} si sceglie una vittima e la si studia 
    \item \textbf{Weaponization:} si mette un trojan che permette accesso remoto in un payload consegnabile (email, USB, web)
    \item \textbf{Sfruttamento:} il codice malevolo viene attivato per portare a termine il suo scopo
\end{itemize}



\section{Differenza tra virus metamorfico e polimorfico}
\subsection{Classificazione dei virus}
È possibile classificare i virus in base alle \textbf{tecniche usate per superare i controlli 
di sistema:}
\begin{itemize}
    \item \textbf{Cifratura del virus:} crea una chiave per \textit{"crittografarsi"}; quando viene 
    chiamato un programma infetto, con tale chiave viene decifrato il virus. Per evitare pattern 
    di bit, durante la propagazione la chiave viene cambiata
    \item \textbf{Stealth virus:} si nasconde dal rilevamento da parte dell'antivirus, tramite mutazione 
    o compressione del codice 
    \item \textbf{Polymorphic virus:} durante la replica crea copie che svolgono la stessa funzione 
    ma che hanno pattern di bit diversi 
    \item \textbf{Metamorphic virus:} si riscrive completamente ad ogni iterazione per aumentare la 
    difficoltà di rilevamento
    \item \textbf{Compression virus:} comprimono il file eseguibile in modo che la versione infetta 
    abbia la stessa dimensione di quella originale
\end{itemize} 



\chapter{Autenticazione}



\chapter{Attacchi TCP}






\chapter{Attacchi}




\chapter{Scanning}



\chapter{Network Scanning}





\chapter{Firewall e NIDS}




\end{document} 