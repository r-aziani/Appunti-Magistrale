\chapter{IDS e IPS}

\section{IDS (Intrusion Detection System)}

Dato che i firewall non possono proteggere da tutte le minacce e un \textit{instrusione}
non è un evento impossibile, è fondamentale e rilevarla e porre rimedio; viene 
fatto tramite gli \textbf{Intrusione Detection Systems}:
\begin{itemize}
    \item \textbf{sistema per identificare individui/servizi che usano un computer o una 
    rete senza autorizzazione}; il controllo è esteso anche ad utenti autorizzati, ma 
    che \textbf{non rispettano i loro privilegi}
    \item viene fatto monitorando il traffico di rete; può collaborare con il firewall
    \item è utile anche per fornire informazioni utili su intrusioni avvenute, fare diagnosi e correggere debolezze
\end{itemize} 

\noindent È importante ricordare che gl'IDS \textbf{non è un sistema di protezione 
ma di rilevazione} delle intrusioni o di attacchi provenienti dall'esterno.

\subsubsection{Caratteristiche funzionali}
\begin{itemize}
    \item \textbf{IDS passivi:}
    \begin{itemize}
        \item checksum crittografici 
        \item riconoscimento di pattern 
    \end{itemize}
    \item \textbf{IDS attivi:}
    \begin{itemize}
        \item \textit{learning} $\rightarrow$ analisi statistica del sistema 
        \item \textit{monitoring} $\rightarrow$ analisi attiva del traffico di dati, azionim \dots
        \item \textit{reaction} $\rightarrow$ confronto con parametri statistici (reazione scatta al superamento di una soglia)
    \end{itemize}
\end{itemize}

\subsection{Caratteristiche topologiche}

\begin{itemize}
    \item \textbf{HIDS (host-based IDS):}
    \begin{itemize}
        \item analisi dei log 
        \item attivazione di strumenti di monitoraggio interni al S.O.
    \end{itemize}
    \item \textbf{NIDS (network-based IDS):}
    \begin{itemize}
        \item attivazione di strumenti di monitoraggio del traffico di rete 
    \end{itemize}
\end{itemize}

\subsubsection{Componenti di un NIDS}
\begin{itemize}
    \item \textbf{Sensor:} controlla il traffico e log per individuare pattern sospetti; attiva i 
    security event quando necessario
    \item \textbf{Director:} coordina i sensor 
    \item \textbf{IDS message system:} consente la comunicazione sicura tra i componenti dell'IDS
\end{itemize}



\subsection{Valutare un IDS}
Per valutare l'efficienza di un IDS occorre conoscere due parametri:
\begin{itemize}
    \item \textbf{Accuratezza:} $allarmi Corretti / allarmi Totali$
    \item \textbf{Completezza:} $allarmi Corretti / intrusioni Totali$
\end{itemize}

\noindent Occore sempre bilanciare i \textit{falsi negativi}
e i \textit{falsi positivi}, dato che questi parametri sono correlati inversamente.

\section{IPS (Intrusion Prevention System)}

La prevenzione è intesa come \textit{velocizzazione dei tempi di risposta} 
una volta che si individua un attacco; il problema relativo a questa tecnologia 
è il rischio di prendere decisioni sbagliate in automatico o il bloccaggio di 
traffico innocuo.

\subsubsection{Honey Pot}

Viene realizzata una rete \textbf{\textit{finta}} per \textbf{attirare gli attacchi}:
quando un attaccante vuole entrare solitamente usa falle note, l'idea è di realizzare 
una rete ad hoc per osservare come gli attacchi vengono effettuati; si realizza una 
DMZ proprio a questo scopo.
\noindent Di norma si mettono moltissimi sensori IDS per misurare tutto 
quello che avviene.

\section{Snort}
Snort è un IDS \textit{open source} molto noto e testato; è in grado di funzionare 
sia in \textbf{modalità sniffer} (osserva il traffico e logga) sia in 
\textbf{modalità IDS} (in questo caso viene regolato con dei file \textit{.rules}).

\noindent Gli alert che vengono sollevati possono essere stampati come \textit{tcpdump}
per salvare spazio e velocizzare la scrittura; $\rightarrow$ Snort è in grado di 
leggere \textit{tcpdump} per fare analisi post-attacco.

\noindent Sono previste quattro \textit{modalità di utilizzo}:
\begin{itemize}
    \item \textbf{Sniffer mode:} legge i pacchetti dalla rete e li visualizza come un flusso continuo 
    \item \textbf{Packet logger mode:} logga i pacchetti salvandoli sul disco 
    \item \textbf{Network instrusion detection system mode:} funziona da NIDS; utilizza regole 
    \textit{snort-based} per individuare pattern di traffico sospetti 
    \item \textbf{Inline mode:} funziona di IPS; riceve i pacchetti direttamente da iptable e collabora per 
    bloccare il traffico sospetto.

    \noindent I sensori di Snort sono integrati nel firewall per fare controlli su iptables (nulla vieta di mettere questo 
    sistema sugli host ma è sconsigliato, richiede hw performante, poco conveniente).
    \noindent Quando è in \textit{inline mode} Snort effettua tre azioni:
    \begin{itemize}
        \item \textit{drop:} ordina a iptables di eliminare pacchetti sospetti 
        \item \textit{reject:} oltre a quanto previsto da \textit{drop}, fa inviare un \textit{tcp\_reset} per terminare la connessione 
        \item \textit{sdrop:} fa eliminare il pacchetto senza loggare
    \end{itemize}

    \noindent Alle volte è meglio usare un \textit{drop} rispetto a un \textit{tcp\_reset} per non dare nessuno strumento di analisi 
    all'attaccante, in modo che non possa trarre alcuna informazione sul sistema.
\end{itemize}











