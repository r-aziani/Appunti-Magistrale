\chapter{Introduzione}

\section{Terminologia}

Un \textbf{sistema} può essere visto come (anche una combinazione di):
\begin{itemize}
    \item \textit{hw}
    \item \textit{sw}
    \item persone che lavorano con \textit{hw} e \textit{sw}
    \item clienti
\end{itemize}

\noindent Un \textbf{attore} può essere:
\begin{itemize}
    \item una \textit{persona} che \textit{interagisce} con il sistema 
    \item un \textit{dispositivo} che \textit{interagisce} con il sistema 
    \item un \textit{ruolo} (cliente)
    \item un \textit{ruolo complesso} (Alice che finge di essere Bob)
\end{itemize}

\noindent Una rete è una configurazione di individui interconnessi. Una \textbf{rete di 
computer} può essere vista sotto due punti di vista:
\begin{itemize}
    \item \textit{\textbf{Fisico:}} una infrastruttura \textit{hw} che connette diversi dispositivi 
    \item \textit{\textbf{Logico:}} un sistema che facilita lo scambio di informazioni 
    tra applicazioni che non condividono uno spazio di memoria
\end{itemize}

\newpage
\section{Sicurezza}
La sicurezza può essere intesa come il \textbf{raggiungimento di un obiettivo in presenza 
di un attacco}; è difficile da assicurare perché l'obiettivo è \textit{negativo}:
\begin{itemize}
    \item \textit{dimostrare che Alice può accedere ad un file è facile}
    \item \textit{dimostrare che nessuno oltre ad Alice può accedervi è molto più 
    difficile}
\end{itemize}

\noindent Di norma si raggiunge con un processo \textbf{iterativo}:
\begin{itemize}
    \item si cerca di trovare l'\textit{anello debole} nel sistema 
    \item si adottano delle \textit{contromisure}
    \item si continua a fare \textit{analisi} in cerca di nuove vulnerabilità
\end{itemize}

\noindent Il concetto di \textit{sicurezza perfetta} non è raggiungibile; per discutere
di sicurezza si deve definire:
\begin{itemize}
    \item \textbf{Politica di sicurezza:} definizione di regole di sicurezza che il sistema 
    deve rispettare
    \item \textbf{Modello di minaccia:} assunzioni su cosa possa fare l'avversario per penetrare
    nel sistema; devo comprendere la potenza dell'avversario
    \item \textbf{Meccanismi:} \textit{sw} o \textit{hw} che cercano di assicuare che la politica sia 
    rispettata, finché l'attaccante segue il modello di minaccia
\end{itemize} 

\noindent Le reti di computer sono sistemi insicuri: abbiamo un sistema complesso (\textit{computer}) in 
un sistema complesso (\textit{rete}) $\rightarrow$ è difficile prevedere da quale punto 
arriveranno gli attacchi e quali vettori verranno sfruttati.

\noindent Ad oggi, le motivazioni dietro agli attacchi sono principalmente:
\begin{itemize}
    \item economiche 
    \item politiche / militari 
    \item attivismo
\end{itemize}


















