\chapter{Cloud Privacy Issues}
Garantire la sicurezza significa garantire $\mathbb{C}$onfidenzialità, $\mathbb{I}$ntegrità e $\mathbb{A}$ccesso (=Disponibilità) ai dati. Nello scenario cloud tre sono i momenti chiave:
\begin{itemize}
    \item memorizzazione dei dati
    \item gestione dei dati
    \item processazione dei dati
\end{itemize}
\section{Introduzione}
Usare il cloud è conveniente per benefici come ad esempio la scalabilità e l'elasticità, però comporta anche dei risbolti problematici perchè il proprietario dei dati non ne ha più il pieno controllo, con conseguente minaccia per la sicurezza che può minare la fiducia nell'adozione del cloud computing. Il NIST distingue 4 modelli di sviluppo del Cloud:
\begin{enumerate}
    \item \textit{privato}: mantenuto su una rete privata
    \item \textit{pubblico}: una organizzazione offre servizi cloud a terzi
    \item \textit{commmunity}: infrastruttura condivisa da diverse compagnie ma con interessi comuni
    \item \textit{ibrido}: un cloud composto da una combinazione di cloud che siano $1$,$2$ oppure $3$
\end{enumerate}
Sempre lo stesso NIST identifica 3 modelli di servizio:
\begin{enumerate}
    \item \textit{IaaS}: \textbf{Infrastructure} as a Service
    \item \textit{PaaS}: \textbf{Platform} as a Service
    \item \textit{SaaS}: \textbf{Software} as a Service
\end{enumerate}
\section{$\mathbb{CIA}$ nel Cloud}
I problemi di sicurezza possono essere classificati con il paradigma $\mathbb{CIA}$. $\mathbb{C}$ è garantire riservatezza delle informazioni salvate esternamente, $\mathbb{I}$ è garantire l'autenticità dei dati, $\mathbb{A}$ richiede che il provider soddisfi dei livelli di servizio attesi.\\
In uno scenario complesso si richiede l'esecuzione di query sui dati e comunque operazioni in cui entra in gioco anche un rapporto di fiducia con il cloud provider (o fornitore) che può essere \textbf{completamente affidabile} (assunto in caso di cloud privati), \textbf{curioso} (archiviazione ed elaborazione coinvolgono informazioni sensibili), \textbf{pigro} (ovvero potenzialmente non affidabile nel caso in cui debba gestire info sensibili), \textbf{dannoso} o bizantino (fornitore si comporta in modo improprio nella gestione, archiviazione ed elaborazione dei dati compromettendo $\mathbb{CIA}$)
\section{Problemi e sfide}
Non vi è una ricetta come soluzione per preservare $\mathbb{CIA}$, però nonostante i differenti aspetti si possono considerare diversi scenari:
\subsection{Protezione dei dati al rilascio}
Un problema dei dati rilasciati nel cloud è garantire la loro protezione (ovvero rispettare la CIA); solitamente gli utenti devono fidarsi completamente dei cloud provider. Questi garantiscono riservatezza da esterni, ma non vi è nessuno che garantisca che il proprietario del cloud possa accedere ai miei dati personali salvati.
\begin{itemize}
    \item Una possibile soluzione potrebbe essere allora la \textbf{cifratura dei dati prima} di rilasciarli ai cloud provider. La cifratura quindi garantisce $\mathbb{C}$ e $\mathbb{I}$. Si preferisce la critto simmetrica. Ma usare la critto presenta problemi quando si vuole fare un \textit{recupero fine dei dati}.
    \item Si adotta allora la frammentazione, al posto della critto, che protegge da associazioni di dati:
    \begin{itemize}
        \item dividendo le parti di informazioni
        \item salvandole in frammenti separati non collegabili
    \end{itemize}
    \item Si può optare anche, nella versione \textit{two can keep a secret}, per fare sì che i frammenti separati siano proprio due provider non comunicanti tra loro, ciascuno dei quali detiene memorizzata una porzione di dati.
    \item Talvolta vengono cifrati solamente gli attributi sensibili oppure non si cifra nulla se vi è fiducia nei confronti del provider.
\end{itemize}
$\mathbb{A}$ è invece garantita dalla replicazione dei dati su più cloud provider.

\subsection{Accesso a grana fine ai dati nel cloud}
Mantenere i dati cifrati nel cloud impedisce la decifrazione da parte dei proprietari per eseguire query. Per realizzare questo approccio è necessario però operare query su dati cifrati, come ad esempio
\begin{itemize}
    \item la \textbf{crittografia omomorfica}: questa però ha svantaggi come ad es:
        \begin{itemize}
            \item realizzabilità solo su operazioni basilari
            \item elevato costo computazionale
        \end{itemize}
    \item \textbf{CryptDB}: supporta maggiori operazioni e migliora l'efficienza rispetto alla omomorfica. In questo modello ogni relazione è cifrata a livello di colonna con diversi livelli di cifratura, ognuno dei quali supporta l'esecuzione di una specifica operazione SQL.
    Funzionamento: quando server riceve la query, determina il livello di profondità di cifratura e il proxy invia al provider la chiave di quel livello per rimuovere i livelli soprastanti di cifratura.
    \item Un altro approccio consiste nell'\textbf{allegare indici} ai dati cifrati per recuperare informazioni a grana fine ed eseguire query. L'indicizzazione può essere \textit{diretta} (1:1) oppure \textit{hash} (N:1) oppure \textit{piatta} (N:1). Queste tre tecniche forniscono diverse garanzie di protezione. Quella hash e piatta garantiscono migliore riservatezza.
\end{itemize}

\subsection{Accesso selettivo ai dati nel cloud}
Idea è che differenti utenti possano avere differenti viste dei dati. Si parla allora della creazione di vere e proprie ACL, spesso però sono delegate al provider (che medierà ogni richiesta di accesso) e questo richiede fiducia da parte del cliente. Se non vi è totale fiducia si può combinare ACL con crittografia con chiavi diverse, a seconda delle autorizzazioni che si detengono all'atto della richiesta di accesso.
Una possibile soluzione al problema del cambio di autorizzazioni nelle ACL è dato dalla \textit{sovracrittografia}.\\
Inoltre, per alleggerire il carico computazionale, si può operare crittografia basata su attributi (ABE), critto a chiave pubblica che regola l'accesso ai dati in base agli attributi descrittivi associati ai dati stessi.
\subsection{Privacy dell'utente}
Idea è di consentire l'accesso ai dati a utenti non registrati nel sistema, cioè senza che dichiarino la propria identità. L'accesso deve essere quindi basato sulle proprietà degli utenti e non sull'identità. Tale accesso può essere basato su:
\begin{itemize}
    \item attributi
    \item credenziali
    \item certificati
\end{itemize}
\subsection{Privacy delle query}
In alcuni scenari non sono privati solo i dati o gli utenti, ma anche gli accessi che questi ultimi fanno per accedere ai dati. Una soluzione è Private Information Retrieval, che però ha un alto costo computazionale. Idea è prelevare l'i-esimo bit di una stringa senza rivelare al server quale bit si sia prelevato. Una soluzione meno onerosa è \textbf{RAM Oblivious}: \textit{ORAM} e \textit{PathORAM}, basati su una struttura dati \textit{gerarchica} e \textit{dinamica} basata su chiave (detta \textit{indice Shuffle}) modifica dinamicamente (shuffling) ad ogni accesso la posizione fisica dei dati, distruggendo la corrispondenza statica dati-blocchi fisici di memorizzazione. Viene creato un B+ albero contiene nodi con blocchi effettivi e blocchi fittizi per creare confusione al provider affinchè non rintracci il blocco mirato. L'indice di Shuffle predispone $n$ ricerche false per non fare capire quale elemento si sia cercato veramente.
\subsection{Integrità della computazione e delle query}
In caso di mancata fiducia con il provider, il cliente deve poter verificare che il risultato di una query sia \textit{corretto} (= risultato calcolato sui dati originari), \textit{completo} (= non mancano dati dal risultato) e \textit{fresco} (= query è stata eseguita sulla versione più recente dei dati).\\
Esistono soluzioni:
\begin{itemize}
    \item \textbf{deterministiche}: strutture dati autentiche come ad es. schemi di concatenamento delle firme (consentono la verifica di ordinamento tra le tuple $\xrightarrow{}$ verifica dell'integrità delle query) e Merkle hash tree (organizza i dati in una struttura ad albero basandosi su un attributo con cui si verifica il risultato della query).
    \item \textbf{probabilistiche}: come ad esempio l'inserimento di \textit{tuple false} nei dati, \textit{replica} di dati, \textit{pre-calcolo dei token} associati ai risultati di una query.\\
    La probabilità di trovare una compromissione dell'integrità dipende dalla quantità di controlli applicati.\\
    Una possibile soluzione per la valutazione dell'integrità dei join calcolati da un provider non attendibile è usare server di archiviazione per inserire informazioni di controllo    
\end{itemize}
\subsection{Esecuzione collaborativa delle query con provider multipli}
Scenario in cui ci sono più provider e anche più proprietari dei dati. Importante è garantire che le informazioni non vengano rese accessibili, rilasciate o trapelate: occorre un rilascio selettivo dei dati. [...]
\subsection{SLA e Auditing}
SLA è un accordo contrattuale che specifica le performance e disponibilità che un provider deve mantenere. In passato SLA era sinonimo di disponibilità, tempo di risposta,...) oggi indica anche criteri di sicurezza (crittografia, protezione perimetrale)
\subsection{Multi-locazione e virtualizzazione}
Multi-locazione significa capacità di fornire servizi informatici a diversi utenti usando infrastruttura cloud comune. Ogni utente dell'infrastruttura cloud condivide risorse di calcolo, come memoria e archiviazione... attività che vengono ben coordinate grazie alla virtualizzazione, che offre grande flessibilità ma introduce problemi di sicurezza come ad esempio \textbf{colpire} l'hypervisor (= sw che crea e coordina le macchine virtuali), riguardare l'\textbf{allocazione/deallocazione} delle risorse. Il rischio è quello di perdere informazioni in modo improprio se le risorse virtuali non sono isolate nel loro assegnamento al singolo utente 
\section{Conclusioni}
La rapida crescita di piattaforme cloud sollecita la considerazione di problemi e preoccupazioni riguardanti la sicurezza in questo scenario, soprattutto per quanto riguarda $\mathbb{CIA}$.