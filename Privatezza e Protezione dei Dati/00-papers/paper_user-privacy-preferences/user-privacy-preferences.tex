\chapter{Supporting User Privacy Preferences in Digital Interactions}
\label{ch.privacy_pref}
%Alessandro
%%%%%%%%%%%%%%%%%%%%%%%%%%%%%%%%%%%%%%%%%%%%%%%%

\section{Introduzione}
Per la disponibilità dei servizi online si verifica la situazione in cui il server che fornisce il servizio e l'utente che lo richiede sono ignoti l'uno all'altro.
Di conseguenza sistemi di access control tradizionali non possono essere adottati in quanto non adatti a scenari open.

Soluzioni proposte sono basate su ABAC (Attribute-Based Access Control).
Policy che regolamentano l'accesso ai servizi definiscono condizioni che il client richiedente deve soddisfare per avere accesso.

Il server, a seguito di una richiesta di accesso risponde con le condizioni che il client deve soddisfare per essere autorizzato ad accedere al servizio.

$\\$
Per accedere il client rilascia \textbf{certificati digitali} come \textbf{credenziali} firmate da una \textbf{CA}.
L'adozione delle credenziali nell'Access Control ha diversi vantaggi:
\begin{itemize}
    \item Non serve $<$\textit{username,password}$>$ per ogni servizio.
    \item Offre migliore protezione dall'acquisizione impropria dei privilegi di accesso.
\end{itemize}

$\\$
L'uso di credenziali è stata approfonditamente studiata dal lato server con definizione di diversi \textit{policy languages}, \textit{policy engines} e strategie per la comunicazione delle consizioni di accesso.

Le soluzioni trovate, sebbene potenti ed espressive, non supportano completamente gli specifici criteri di protezione dei client.
Infatti i client necessitano di un sistema di preferenze che permetta di supportare una propria definizione di sensitivity/privacy per i propri dati.
Queste preferenze verranno poi usate nella situazione in cui più di un subset di credenziali soddisfano la Access Control Policy definita dal server.




Questo capitolo mostra una overview delle problematiche di privacy legate ai sistemi open, oltre che alcune soluzioni.
Le sezioni riportano:
\begin{itemize}
    \item Concetti Base e \textit{Desiderata}.
    \item Soluzione Cost-Sensitive Trust Negotiation. 
    \item V
    \item c
    \item d
\end{itemize}


\section{Concetti Base e \textit{Desiderata}}

\subsection{Client Portfolio}

\begin{definition}[\textit{Client} Portfolio] $\\$
    Insieme delle informazioni che un cliente può fornire al server per avere accesso ad un servizio è racchiuso in un \textbf{Portfolio} contenente \begin{itemize}
        \item \textbf{Credenziali} firmate da third party.
        \item \textbf{Dichiarazioni} che comprendono proprietà non certificate dichiarate dall'utente.
    \end{itemize} 
\end{definition}

\subsubsection{Struttura Credenziale}
Ogni credenziale $c$ è composta da:
\begin{itemize}
    \item Identificatore univoco $id(c)$.
    \item Emittente $issuer(c)$.
    \item Un set di attributi $attributes(c)$.
    \item La tipologia della credenziale $type(c)$.
\end{itemize}


\subsubsection{Gerarchia Tipologie}
I tipi di credenziale sono organizzati tipicamente in una gerarchia radicata, dove i nodi intermedi sono astrazione delle specifiche credenziali presenti sulle foglie. Un esempio è in fig. \ref{fig:00_pref_cred_hierarchy}.

\begin{figure}[ht]
    \centering
    \includegraphics[width=0.8\textwidth]{paper_user-privacy-preferences/00_pref_credential_hierarchy.jpg}
    \caption{}
    \label{fig:00_pref_cred_hierarchy}
\end{figure}

La gerarchia è nota sia al client che al server. Il server tuttavia crea richieste di credential types in quanto non può conoscere le effettive credenziali possedute dal client\footnote{Inoltre client può avere più credenziali dello stesso tipo}.

\subsection{Atomicità Credenziali}
Le credenziali si distinguono in \textbf{atomiche} e \textbf{non-atomiche}.

$\\$
Le credenziali \textbf{atomiche} sono la tipologia più comune (vedi X.509) e possono essere rilasciato solo interamente.
Per questo anche attributi non richiesti vengono rivelati al server.

$\\$
Le credenziali \textbf{non-atomiche} permettono di limitare il rilascio di attributi rivelando selettivamente un subset degi attributi certificati dalla credenziale.

$\\$
Ovviamente le \textit{declaration} sono credenziali non-atomiche.


\subsubsection{Attributi}
Attributi nelle credenziali sono caratterizzati da:
\begin{itemize}
    \item Tipo
    \item Nome
    \item Valore
\end{itemize}
Essi possono essere \textit{credential-independent} e dipendere dal client ma non dalla credenziale, o \textit{credential-dependent} e dipendere da client e credenziale che la certifica.



\subsection{Policy di Disclosure}
Le policy lat server che regolano l'accesso sono definite sugli attributi e le credenziali fornite dall'utente, questo perchè il server considera solo la gerarchia di credenziali (conosciuta a priori anche dal server).

Una Access Control Policy (policy di controllo di accesso) è quindi una espressione booleana composta da condizioni \textit{ $ term_1 \, op \, term_2 $ } , dove \textit{op} è un operatore del tipo $<, >, =$ etc.

Un esempio di policy: \textit{ (type(c)= insurance) $\land$ (c.Company $\ne$ 'A') }





\subsection{Trust Negotiation}
L'interazione tra client e server mira a stabilire una relazione di fiducia (trust) tra le parti per permettere l'accesso al servizio.
Questa interazione si basa tipicamente sul rilascio di credenziali, spesso subordinato al rilascio di altre credenziali da parte della controparte.

Questa sequenza di scambi di credenziali è definita \textbf{strategia} e soddisfa le policy di server e client.

Alcune strategie adottate permettono il rilascio di credenziali solo se esiste una \textit{strategia di successo} (successfull strategy) che garantisce l'accesso al servizio, mentre altre rilasciano credenziali appena sono richieste.

$\\$
Il fatto che esistano preferenze sul rilascio di credenziali, sia da parte del server che del client, comporta che le successful strategies non siano tutte equivalenti. 






\subsection{Client Privacy Preferences}
Per quanto riportato, è necessario fornire i client di un modello flessibile ed espressivo per rappresentare le preferenze di privacy.
Questo modello dovrebbe avere le seguenti caratteristiche:

\begin{itemize}
    \item \textbf{Specific di Preferenze a Grana Fine}: il modello deve supportare la definizione di preferenze per ogni attributo per ogni \textbf{istanza} di credenziale.
    \item \textbf{Ereditarietà delle Preferenze di Privacy}: se le istanze di credenziali sono numerose, il modello deve poter applicare sensibilità ai tipi astratti e così a tutte le istanze per cui non era stato specificato un livello di sensibilità.
    \item \textbf{Rel. d'Ordine Parziale e Operatore Composizione}: ci deve essere ordinamento parziale per determinare se una informazione è più o meno sensibile di un'altra. Deve esistere un operatore di composizione $\bigoplus$ che permette di calcolare il valore di sensibilità di un set di attributi/credenziali.
    \item \textbf{Associazioni Sensibili}: rilascio combinato di un set di attr./cred. risulta più o meno sensibile della somma dei rilasci singoli. Il modello deve comprendere anche questa definizione di preferenza.
    \item \textbf{Vincoli di Rilascio}: modello deve permettere di definire vincoli sul rilascio combinoato di parti del portfolio, per evitare o limitare il rilascio dell'associazione tra questi.
    \item \textbf{Context-Based Preferences}: preferenze dipendono dal contesto (rilasci certificato medico a una farmacia, non a un hotel).
    \item \textbf{History-based Preferences}: rilascio dipende da interazioni passate (se a server X ho rilasciato un set, uso sempre quello e non fornisco nuove cred.).
    \item \textbf{Prova di Possesso}: nuove tecnologie permettono di rilasciare prove di possesso di certificati e di soddisfazione requisiti di accesso. Risulta necessario poter specificare preferenze anche per queste prove (proof).
    \item \textbf{User-Friendly Preference Specification}: necessario fornire interfacce ai client, che possono non essere abituati a sistemi di controllo dell'accesso, per poter specificare facilmente le preferenze.
\end{itemize}




\subsection{Preferenze Privacy del Server}
Mentre la comunicazione della policy server completa favorisce la privacy del client, la comunicazione dei soli attributi coinvolti favorisce la privacy del server.

Ricordare che anche server ha un portfolio di credenziali e attributi oltre alle policy.

Il sistema lato server deve considerare:
\begin{itemize}
    \item \textbf{Disclosure Policy}: server deve poter decidere, a grana fine, come il rilascio policy dovrebbe essere regolato.
    \item \textbf{Comunicazione Policy}: necessario un meccanismo che trasforma la A.C.Policy in modo da proteggere privacy della policy e permettere al client di sapere cosa è richiesto.
    \item \textbf{Integrazione con Meccanismo Client}: Ci deve essere integrazione ra i sistemi lato server e lato client.
\end{itemize}





\section{Cost-Sensitive Trust Negotiation }
Sensibilità espressa con costo $w(c)$ 