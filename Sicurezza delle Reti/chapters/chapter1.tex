\chapter{Standard e Concetti base}

\section{Sicurezza informatica - definizioni}

Ci sono standard internazionali a cui si può fare riferimento quando si parla di sicurezza; utilizzando questi standard si può determinare se un sistema è sicuro o meno.

\begin{itemize}
    \item insieme di approcci, linee guida, strumenti che possono essere utilizzate per proteggere l'ambiente e le risorse dell'organizzazione e degli utenti
    \item i beni dell'organizzazione e degli utenti comprendono i dispositivi connessi, il personale, infrastrutture, ecc. e la totalità delle informazioni trasmesse e/o archiviate nel cyberspazio
    \item la sicurezza informatica si impegna a garantire il raggiungimento e mantenimento delle proprietà di sicurezza dell'organizzazione contro i possibili rischi
\end{itemize}
La sicurezza informatica si può divedere in:
\begin{itemize}
    \item \textbf{Sicurezza delle informazioni:} devono essere rispettate le proprietà come integrità, confidenzialità e disponibilità
    \item \textbf{Sicurezza della rete:} protezione delle reti e del loro servizio da modifiche non autorizzate; garanzia che la rete svolga sempre le sue funzioni correttamente
\end{itemize}

\subsubsection{Le sfide della sicurezza informatica}

\begin{itemize}
    \item La sicurezza non è semplice (requisiti semplici ma meccanismi complessi)
    \item Nello sviluppo di un meccanismo di sicurezza, si devono sempre considerare potenziali attacchi
    \item Decidere dove utilizzare i meccanismi di sicurezza (fisicamente in che punto della rete e logicamente a che livello dell'architettura)
    \item I meccanismi di sicurezza generalmente coinvolgono più di un algoritmo o protocollo
    \item Battaglia tra progettista e attaccante
    \item Percezione di scarsi benifici dall'investemento nella sicurezza (fino a quando non si verifica un errore)
    \item La sicurezza richiede un monitoraggio costante
    \item La sicurezza è troppo spesso a posteriori (dopo che il sistema è stato progettato)
    \item La sicurezza avanzata può rappresentare un impedimento al funzionamento efficiente e di facile utilizzo
\end{itemize}

\begin{figure}[h]
    \centering
    \includegraphics[width=0.95\linewidth]{chapters/images1/sicurezza.png}
    \caption{Concetti di sicurezza}
    \label{fig:sec}
\end{figure}

\begin{itemize}
    \item \textbf{Attacchi} alla sicurezza: qualsiasi azione che comprometta la sicurezza di un'informazione
    \item \textbf{Meccanismi} di sicurezza: un processo progettato per rilevare, prevenire o recuperare da un attacco
    \item \textbf{Servizi} di sicurezza: contrastano gli attacchi, si avvalgono di uno o più meccanismi per fornire il servizio
\end{itemize}

\subsubsection{Attacchi passivi e attivi}
\begin{itemize}
    \item \textbf{Attacchi passivi:} NON alterano le informazioni; lo scopo dell'attacco è ottenere informazioni sui messaggi trasmessi
    \begin{itemize}
        \item accesso al contenuto del messaggio
        \item analisi del traffico di rete (la frequenza o la lunghezza dei messaggi potrebbero rivelare la natura della comunicazione)
    \end{itemize}
    \item \textbf{Attacchi attivi:} modificano il flusso delle informazioni
    \begin{itemize}
        \item fingere di essere qualcun'altro
        \item denial of service
    \end{itemize}
\end{itemize}

\section{Minacce e conseguenze}
Per threat si intende una potenziale violazione della sicurezza.

Le azioni che potrebbero causare una violazione devono essere protette o preparate; queste azione vengono chiamate \textbf{attacchi}.

Le minacce possono essere divise in quattro gruppi:
\begin{itemize}
    \item \textbf{Divulgazione non autorizzata;} è una minaccia alla confidenzialità.
    \begin{itemize}
        \item \textbf{esposizione}: un errore umano, software o hardware conduce alla rivelazione di dati sensibili
        \item \textbf{intercettazione}
        \item \textbf{inferenza}: l'attaccante è in grado di ottenere dalla sola osservazione del traffico
        \item \textbf{intrusione}: un attaccante ottiene accesso a dati sensibili superando un controllo di accesso
    \end{itemize}
    \item \textbf{Inganno;} è una minaccia all'integrità dei dati.
    \begin{itemize}
        \item \textbf{mascheramento:} tentativo da parte dell'attacante di ottenere l'accesso a un sistema fingendosi un utente autorizzato
        \item \textbf{falsificazione:} alterazione di dati validi o inserimento di dati falsi in un database
        \item \textbf{ripudio:} un utente rinnega di aver inviato o ricevuto dei dati
    \end{itemize}
    \item \textbf{interruzione;} è una minaccia alla disponibilità o integrità di un sistema
    \begin{itemize}
        \item \textbf{interdizione:} danneggiamento dell'hardware
        \item \textbf{corruzione:} le risorse funzionano in modo non voluto
        \item \textbf{ostruzione:} interferire con le comunicazioni alterandone i collegamenti
    \end{itemize}
    \item \textbf{usurpazione;} è una minaccia all'integrità del sistema.
    \begin{itemize}
        \item \textbf{approprazione indebita:} ad esempio una sottrazione del servizio (DDOS)
        \item \textbf{uso improprio:} ad esempio dopo che un utente ha ottenuto un accesso non autorizzato  
    \end{itemize}
\end{itemize}

\subsubsection{Superificie di attacco}
È costituita dalle vulnerabilità raggiungibili e sfruttabili in un sistema, come ad esempio:
\begin{itemize}
    \item le porte aperte verso l'esterno
    \item servizi disponibili all'interno di un firewall
    \item codice che elabora dati in entrata
    \item un dipendente con accesso a dei dati sensibili (social engeneering)
\end{itemize}

Alcune superfici di attacco:
\begin{itemize}
    \item \textbf{superificie di attacco di rete}; sono incluse vulnerabilità del protocollo di rete
    \item \textbf{superificie di attacco software}; sono incluse vulnerabilità nel codice delle applicazioni
    \item \textbf{superificie di attacco umano;} sono incluse vulnerabilità create dal personale (errori, social engeneering)
\end{itemize}

