\documentclass{report}
\usepackage{graphicx} % Required for inserting images
\usepackage[italian]{babel}
\usepackage{tikz}
\usepackage{hyperref}
\usepackage{amsmath}
\usepackage{xcolor}

\definecolor{darkgreen}{rgb}{0.0, 0.5, 0.0}


\title{Affidabilità dei sistemi}
\date{Annalisa Pasquali}

\begin{document}

\maketitle

\tableofcontents
\newpage


\chapter{L'affidabilità come requisito della qualità}

\section{Introduzione}
Affinché si possa esprimere un giudizio sul livello qualitativo del
prodotto così ottenuto può essere utile richiamare la definizione
del termine Qualità fornita dalla norma UNI EN ISO 9000 che, a
tale proposito, recita: “Qualità è il grado in cui un insieme di
caratteristiche intrinseche soddisfa i requisiti”.

\noindent \textbf{Dependability:} La norma CEI 56-50 introduce il concetto di fidatezza (dependability), intesa
come \textbf{l’insieme delle proprietà che descrivono la disponibilità e i
fattori che la condizionano cioè affidabilità, manutenibilità e
logistica della manutenzione}.

\section{Requisiti RAMS}
Intesi come
\begin{itemize}
    \item \textbf{Reliability} (Affidabilità): L’affidabilità è definita, in termini
    qualitativi, come “l’attitudine dell’elemento a svolgere la funzione
    richiesta in condizioni date per un dato intervallo di tempo.    
    \item \textbf{Availability} (Disponibilità):attitudine dell’entità ad essere in grado
    di svolgere la funzione richiesta in determinate condizioni a un dato
    istante, o durante un dato intervallo di tempo, supponendo che
    siano assicurati i mezzi esterni eventualmente necessari
    \item \textbf{Maintainability} (1- Manuteniblità e 2-supporto logico alla manutezione): 1-attitudine dell’entità in
    assegnate condizioni di utilizzazione a essere mantenuta o
    riportata in una stato nel quale essa può svolgere la funzione
    richiesta, quando la manutenzione è eseguita nelle condizioni
    date, con procedure e mezzi prescritti. 
    2- attitudine di una organizzazione della
    manutenzione, in date condizioni, a offrire dietro richiesta le
    risorse necessarie alla manutenzione dell’entità, conformemente
    ad una data politica di manutenzione.
    \item \textbf{Safety}(Sicurezza): assenza di rischi inaccettabili e determinabile
    attraverso livelli di integrità alla sicurezza SIL (Safety Integrity
    Level).
    \item + Security(Protezione): nei confronti della prevenzione di accessi non
    autorizzati e/o manipolazioni di informazioni private.
\end{itemize}

\subsection{Affidabilità VS Disponiblità}
Distinzione tra affidabilità e disponibilità, sono correlate ma diverse:
\begin{itemize}
    \item \textbf{Affidabilità:} è la capacità di un sistema o componente di funzionare correttamente per un certo periodo senza guasti. Viene espressa come la probabilità che il sistema esegua la sua funzione entro il tempo utile. Un esempio è un aereo che deve completare il volo senza guasti critici.
    \item \textbf{Disponiblità:} è il grado di funzionamento e accessibilità del sistema quando necessario. È influenzata sia dall'affidabilità sia dal tempo di ripristino in caso di guasto. Per sistemi a lungo funzionamento continuo (come i servizi cloud), la disponibilità è cruciale, poiché i guasti sono inevitabili, e la rapidità di ripristino diventa determinante per minimizzare i tempi di inattività. 
\end{itemize}
La disponibilità è quindi una funzione che tiene conto sia dell’affidabilità
del sistema sia degli aspetti manutentivi.
I problemi di affidabilità possono allora essere trattati come casi
particolari di quelli di disponibilità per i quali il passaggio allo stato di
guasto non consente il ritorno allo stato di funzionamento.

\noindent Quando è più importante l’affidabilità e quando lo è la disponibilità.

- È più importante l’Affidabilità:
Quando è prevalente il costo del guasto (per esempio perché è necessaria
una sostituzione di componenti, una riparazione dei danni provocati, etc…)
è più significativa l'affidabilità.

- È più importante la Disponibilità:
Quando è prevalente il costo connesso con il guasto (per esempio per la
mancata produzione che ne scaturisce, oppure per il mancato servizio,
etc...) allora è più significativa la disponibilità.

\end{document}