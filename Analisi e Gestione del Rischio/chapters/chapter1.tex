\chapter{Introduzione}

Il termine \textit{sicurezza} porta con sé una contraddizione: esso, infatti,
fa venire in mente qualcosa di assoluto e incontrovertibile, cioè qualcosa
di impossibile nella realtà.

Non ha senso parlare di sicurezza in senso assoluto, ma solo in senso relativo.

Deve essere individuato il livello \textit{adeguato} di sicurezza che si vuole
ottenere attraverso la \textbf{valutazione del rischio}. Il livello di sicurezza
deve essere ottenuto attraverso opportune azioni di \textit{trattamento}. Nel caso in cui
il livello non possa essere raggiunto, le carenze devono essere analizzate ed
eventualmente accettate.

Nel tempo, la valutazione deve essere ripetuta per verificare se il livello di sicurezza
desiderato ed attuato siano ancora validi. Queste attività costituiscono la
\textbf{gestione del rischio}.
