\documentclass{report}
\usepackage{graphicx} % Required for inserting images
\usepackage[italian]{babel}
\usepackage{tikz}
\usepackage{hyperref}
\usepackage{amsmath}
\usepackage{xcolor}

\definecolor{darkgreen}{rgb}{0.0, 0.5, 0.0}


\title{Sistemi Biometrici basati sull'Iride}
\date{Parte V}

\begin{document}

\maketitle

\tableofcontents
\newpage


\chapter{Caratteristiche dell'iride e creazione dell'Iriscode}

L'iride è considerato il tratto biometrico più accurato
dopo il DNA. È poco gradito dagli utenti per la sua
"percepita" invasività.

\noindent L'iride presenta caratteristiche numerosissime e stabili 
nel tempo; inizia a crearsi dal terzo mese nel feto, al settimo mese 
il processo è completato ma diventa stabile dal secondo anno di vita.

\noindent Il sistema è piuttosto complesso e costoso, ma difficile da frodare.

\section{Vantaggi e svantaggi}

Tra i \textbf{\textcolor{green}{vantaggi}} troviamo:
\begin{itemize}
    \item acquisizione senza contatto
    \item molte caratteristiche casuali e distintive
    \item il tratto appartiene ad un organo interno generalmente protetto e presente in tutta la popolazione
    \item esistono sistemi di acquisizione ed elaborazione molto veloci
\end{itemize}
Tra gli \textbf{\textcolor{red}{svantaggi}} troviamo:
\begin{itemize}
    \item l'acquisizione può essere difficile 
    \item l'iride è piatta ma è dietro una superficie curva e bagnata (cornea)
    \item una buona parte è nascosta da ciglia e palpebra
    \item si deforma con la dilatazione della pupilla
    \item invecchiando possono comparire delle pigmentazioni non presenti
    precedentemente
\end{itemize}

\section{Struttura dell'iride}
L'iride è una \textbf{membrana piatta che sta tra la cornea e il cristallino.}
Ha il compito di controllare il livello di intensità luminosa che deve entrare nell'occhio.

La \textbf{cornea} è una struttura trasparente a forma di cupola 
posizionata nella parte anteriore al centro della sclera (parte bianca).

Il \textbf{cristallino} è una lente elastica, trasparente, le cui contrazioni muscolari ne 
permettono l'ispessimento o restringimento per consentire all'occhio di mettere a fuoco oggetti 
posti a distanze diverse.

\begin{figure}[ht]
    \centering
    \includegraphics[width=0.4\linewidth]{images/struttura-iride.png}
\end{figure}

\section{Unicità dell'iride}

Come nel caso delle impronte, \textbf{non esistono due iridi uguali}.

\noindent Durante la formazione dell'iride si hanno delle componenti casuali che 
producono un pattern di righe, tagli e pieghe (le feature iridee) \textbf{unico
e distinguibile}.

\noindent Anche i gemelli omozigoti hanno iridi diverse.

\begin{figure}[ht]
    \centering
    \includegraphics[width=1\linewidth]{images/unicita.png}
\end{figure}

\noindent Le feature più interessanti per il sistema biometrico si \textbf{vedono meglio con luce IR} 
piuttosto che la luce visibile.

\noindent È necessario inoltre usare \textbf{telecamere con ottiche variabili} per trovare
l'occhio nel volto e poi zoomare verso l'occhio per acquisirlo alla massima
risoluzione possibile.

\section{Rappresentazione delle iridi}
I moderni sistemi per il riconoscimento basati sull'iride rappresentano l'iride come 
una stringa di bit, chiamata \textbf{iriscode}.

\noindent I passi che permettono di passare da una immagine di un occhio ad un iriscode sono:
\begin{enumerate}
    \item individuazione dei centri e raggi della pupilla e dell'iride 
    \item rimozione della parte non utile occupata da ciglia 
    \item linearizzazione dell'iride
    \item trasformazione dell'iride linearizzata in wavelet
    \item trasformazione della trasformata wavelet in bit, ovvero iriscode
\end{enumerate}

\subsection{Calcolo dei centri e raggi di iride e pupilla}
Viene trovato l'occhio nell'immagine del volto, successivamente la pupilla 
e il raggio esterno dell'iride.

\subsection{Rimozione di palpebre e ciglia}
Solo una parte dell'iride è utile al riconoscimento; occorre segmentare 
solo la parte utile dell'iride.

\noindent Se manca più del 50\% dell'iride occorre riacquisire l'immagine.

\subsection{Linearizzaione dell'iride}
Per aumentare l'efficienza degli algoritmi, l'iride non viene elaborata
direttamente nell'immagine in ingresso ma convertita in un rettangolo.

\noindent Questa operazione viene fatta passando da delle coordinate cartesiane $(x, y)$ a polari $(r, 0)$.
\begin{figure}[ht]
    \centering
    \includegraphics[width=1\linewidth]{images/linear-1.png}
\end{figure}

\begin{enumerate}
    \item vengono individuati i raggi e centri
    \item si fissano le dimensioni dei settori dell'iride, 
    scegliendo numero di corone e ampiezza dell'angolo di scansione
    \item ogni pixel del \textit{rubber sheet model} è la media del colore del relativo settore
\end{enumerate}


\begin{figure}[ht]
    \centering
    \includegraphics[width=0.5\linewidth]{images/linear-2.png}
\end{figure}

\newpage
\section{Proprietà dell'iris code}

\begin{itemize}
    \item La codifica dell'iride avviene in uno spazio 2D, rendendo la codifica \textbf{invariante}
    rispetto a:
    \begin{itemize}
        \item dimensione dell'iride
        \item zoom dei sistemi ottici
        \item e dilatazione dell'iride
    \end{itemize}
    \item È \textbf{invariante} rispetto a:
    \begin{itemize}
        \item contrasto dell'immagine 
        \item livello di grigio medio nell'immagine 
        \item illuminazione
    \end{itemize}
    \item È \textbf{molto compatto}, di solito bastano 256 byte per rappresentare una iride, più altri 
    256 byte di controllo per escludere i bit dovuti ad artefatti (\textit{riflessi sulla cornea, ciglia, palpebre, troppo poco contrasto})
    \item La probabilità di ogni bit di essere 1 è pari al 50\%
    
    $\rightarrow$ l'iriscode è un codice a \textbf{massima entropia}
\end{itemize}

\chapter{Algoritmi di enhancement e prefiltraggio}

\section{Controllo qualità del sample}
\begin{figure}[ht]
    \centering
    \includegraphics[width=1\linewidth]{images/prefiltraggio.png}
\end{figure}


\subsection{Focus assessement}
La probabilità di avere una iride già a fuoco è molto bassa, perché:
\begin{itemize}
    \item ingrandimento del sistema ottico 
    \item l'occhio si muove rapidamente (micromovimenti non controllabili)
    \item illuminazione limitata
\end{itemize}

\noindent Per questo il fuoco dell'iride viene controllato con dei filtri
software.\\


\noindent L'effetto di sfocatura dovuto ad un sistema ottico può essere parzialmente riconvertito
usando un \textbf{filtro di deblur}, che si comporta al "contrario" della parte 
che produce la sfocatura:
\begin{itemize}
    \item tende ad aumentare le frequenze alte dell'immagine che erano state attenuate dal blur 
    \item può essere seguito a livello di frame (le prestazioni lo consentono)
\end{itemize}

\subsection{Acquisizioni fuori fuoco}
Se l'immagine è troppo mossa occorre scartarla, in quanto l'iriscode risultante 
sarebbe composto solo da bit casuali dipendenti dal rumore 

$\rightarrow$ se fosse eseguito un enroll con un template simile si avrebbe sicuramente un false non-match

\noindent Il controllo equivale a verificare se è presente una sufficiente quantità di
"dettagli fini" maggiore di una determinata soglia.

\section{Contrast Adjustment}
Non si usano particolari algoritmi di prefiltraggio nel caso dell'iride;
in alcuni casi si applica un \textbf{constrast stretching} per migliorare il contrasto 
e quindi la leggibilità dell'immagine.

\newpage
\section{Modulo Estrazione delle Feature}

\begin{figure}[ht]
    \centering
    \includegraphics[width=1\linewidth]{images/estrazione-features.png}
\end{figure}

\subsection{Come trovare l'iride}
Si cercano nell'immagine i centri di contorni di variazioni di grigio di forma circolare.

\begin{figure}[ht]
    \centering
    \includegraphics[width=0.6\linewidth]{images/ricerca-iride.png}
\end{figure}

\subsubsection{Estrazione delle palpebre}
\noindent Modificando gli operatori della formula è possibile trovare le palpebre, 
immaginandole come parte di una parabola.

\begin{figure}[ht]
    \centering
    \includegraphics[width=0.5\linewidth]{images/estrazione-palpebre.png}
\end{figure}



\subsubsection{Estrazione delle ciglia}
Le ciglia vengono individuate sfruttando la conoscenza che:
\begin{itemize}
    \item devono partire dalla posizione individuata delle palpebre 
    \item risultano sempre più scure dell'iride 
    \item la loro larghezza è facilmente stimabile
\end{itemize}

\noindent È molto importante \textbf{individuare ogni parte dell'immagine che 
non sia iride}, per \textbf{non inserire in enroll una informazione errata}.

\begin{figure}[ht]
    \centering
    \includegraphics[width=0.3\linewidth]{images/estrazione-ciglia.png}
\end{figure}

\chapter{Algoritmi di matching e prestazioni}

\begin{figure}[ht]
    \centering
    \includegraphics[width=1\linewidth]{images/intro-chap3.png}
\end{figure}

\section{Distribuzione dei bit nell'iriscode}
La probabilità che un bit sia 1 oppure 0 in un iriscode è 
intorno al 50\%, rendendolo un codice a \textbf{massima entropia}.

\noindent Sono presenti tuttavia delle correlazioni fra i bit all'interno 
dell'iriscode:
\begin{itemize}
    \item la struttura è auto-predittiva; alcuni gruppi di bit portano ad incontrare nell'iriscode 
    con maggiore probabilità altri gruppi di bit 
    \item viene limitato il grado di libertà di un iriscode, non è "del tutto casuale"
\end{itemize}

\section{Alcuni fattori che aumentano la variabilità intraclasse}

\begin{figure}[ht]
    \centering
    \includegraphics[width=0.9\linewidth]{images/variabilita-intraclasse.png}
\end{figure}

\section{Algoritmi di matching}
La comparazione viene effettuata su iriscode di 256 byte, attraverso 
il calcolo della \textbf{distanza di Hamming}, che conta i bit 
in disaccordo fra le due stringhe.

\noindent In teoria, due iriscode calcolati dall'iride della stessa persona
dovrebbero avere la distanza di Hamming = 0:
\begin{itemize}
    \item in realtà, fra due acquisizioni eseguite perfettamente a pochi 
    istanti una dall'altra si possono osservare delle differenze

    $\rightarrow$ vengono fatte 3 comparazioni e si sceglie quella con valore minimo (della distanza)
    \item nella seconda e terza comparazione, i bit di uno dei due template vengono shiftati rispettivamente a sinistra e poi a destra
\end{itemize}

\section{Maschere}

\subsection{Creazione delle maschere}
Se vi sono delle occlusione dell'iride (palpebre, riflessi, ciglia) occorre 
preparare delle maschere di oscuramento delle zone dove non vi è 
informazione utile.

\begin{figure}[ht]
    \centering
    \includegraphics[width=0.5\linewidth]{images/creazione-maschere.png}
\end{figure}

\subsection{Uso delle maschere}

Si va a togliere dalla distanza di Hamming le zone che non 
devono essere cosiderate.

\section{Velocità del matching}
L'esecuzione degli AND e XOR può avvenire a blocchi di bit pari 
alla lunghezza di parola del processore, rendendo il processo 
di matching veloce.

\newpage
\section{Distribuzione del match score}

\begin{figure}[ht]
    \centering
    \includegraphics[width=1\linewidth]{images/distr-match-score.png}
\end{figure}

\section{Svantaggi della tecnica}





\end{document}