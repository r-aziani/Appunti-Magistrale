\documentclass{report}
\usepackage{graphicx} % Required for inserting images
\usepackage[italian]{babel}
\usepackage{tikz}
\usepackage{hyperref}
\usepackage{amsmath}
\usepackage{xcolor}
\usepackage{float}
\usepackage{soul}
\usepackage{listings} % Per evidenziare il codice

\definecolor{lightgray}{rgb}{0.9,0.9,0.9} % Definizione colore sfondo
\definecolor{darkgreen}{rgb}{0.0, 0.5, 0.0}

\lstset{
    backgroundcolor=\color{lightgray}, % Sfondo grigio
    basicstyle=\ttfamily, % Font monospaziato
    % frame=single, % Bordo attorno al codice
    tabsize=4, % Dimensione tabulazione
    breaklines=true, % Permette di andare a capo automaticamente
    numbers = left,
    numberstyle=\small\color{gray}
}

\title{\huge\textbf{{Domande}}}
\date{Parte extra}
\begin{document}

\maketitle
\tableofcontents
\newpage

\chapter{Domande pacchetto 1 - AC}
\begin{itemize}
    \item Nell'ambito delle politiche basate sui ruoli, descrivere il principio di separazione dei privilegi (statico e dinamico)
e fornire un esempio per entrambe le tipologie.
    \item Nell'ambito delle politiche discrezionarie, dire cosa rappresentano le ACL e le capability. Illustrare i vantaggi
e gli svantaggi.
    \item Nell'ambito del modello relazionale multilivello, dire cosa si intende per tuple poliinstanziate ed elementi
poliinstanziati e fornire un esempio per entrambi i concetti.
    \item Descrivere, tramite un esempio, perchè le politiche discrezionarie sono vulnerabili a Trojan horse.
    \item Nell'ambito del controllo dell'accesso, definire il concetto di gruppo e ruolo, evidenziando quali sono le differenze
tra questi due concetti.
    \item Nell'ambito del modello di Biba, descrivere la politica low-water mark per oggetti. Questa politica garantisce
l'integrità delle informazioni? Si richiede di giustificare la risposta.
    \item Definire il concetto di politica aperta e politica chiusa. Per quale ragione può essere utile definire un sistema
basato sia su autorizzazioni positive sia su autorizzazioni negative? Quali problemi possono sorgere a causa
della presenza di autorizzazioni positive e negative?
    \item Descrivere il concetto di poliinstanziazione. Fare un esempio di tabella con tuple e con elemento poliinstanziati.
    \item Nell'ambito del modello relazionale multilivello, dire cosa si intende per tuple poliinstanziate ed elementi poliinstanziati
e fornire un esempio per entrambi i concetti. Si richiede di descrivere perchè nasce la poliinstanziazione
e la differenza tra visibile ed invisibile.
    \item Descrivere cosa si intende per relation profile nel modello di autorizzazione che prevede tre livelli di visibilità (plaintext, encrypted, no visibility).
    \item Si richiede di fornire la definizione di politica di risoluzione dei conflitti “most specific takes precedence” e “most
specific along a path takes precedence”. Inoltre si richiede di fornire un esempio di gerarchia utenti-gruppi e
di autorizzazioni in modo tale che l'utente alice possa leggere il file file1 quando si applica la politica “most
specific takes precedence” per risolvere i conflitti mentre deve rimanere il conflitto quando si applica la politica
“most specific along a path takes precedence” (usare una sola gerarchia utenti-gruppi per mostrare
queste due situazioni).
    \item Nell'ambito del modello a matrice di accesso, si richiede di fornire la definizione di copy flag e di transfer-only
flag. Fornire inoltre un esempio di comando per entrambi i flag e che riguardi il privilegio di scrittura.
    \item Cosa si intende per sistemi di controllo degli accessi aperti, chiusi ed ibridi?
    \item Nell'ambito dei modelli per la specifica di autorizzazioni basati su tre livellli di visibilità (plaintext, encrypted
e no visibile), cosa cattura il profilo di una relazione? Quale è il profilo di una relazione $R$ ottenuta tramite il prodotto cartesiano di due relazioni $R_1$ e $R_1$?
    \item Si richiede di descrivere le caratteristiche principali della politica per il controllo dell'accesso MAC (Mandatory Access Control).
    \item Nell'ambito delle politiche di controllo dell'accesso DAC, si richiede di descrivere le principali debolezze di queste politiche (trojan horse).
\end{itemize}

\chapter{Domande pacchetto 2 - Query}
\begin{itemize}
    \item Nell'ambito delle tecniche usate per verificare l'integrità del risultato di query, dire quale è la differenza tra
tecniche deterministice e tecniche probabilistice. Si richiede inoltre di fornire un esempio di query e come si
può utilizzare una tecnica deterministica per verificarne il risultato della query di esempio.
    \item Nell'ambito delle tecniche per l'integrità del risultato di query, si richiede di descrivere l'approccio basato sul
Merkle-tree e di fornire un semplice esempio di verifica, specificando la query, la tabella su cui viene eseguita
eil Merkle tree costruito sulla tabell.
    \item Nell'ambito delle tecniche per la verifica della integrità del risultato di query, si richiede di descrivere le differenze
principali tra le tecniche deterministiche e le tecniche probabilistiche. Si richiede inoltre di fare un esempio di
tecnica probabilistica e del suo funzionamento.
    \item Nell'ambito delle tecniche usate per verificare l'integrità del risultato di query, dire quale è la differenza tra
tecniche deterministiche e tecniche probabilistiche. Si richiede inoltre di fornire un esempio di query e come si
usa una tecnica deterministica per verificarne il risultato.
    \item Nell'ambito delle problematiche di integrità del risultato di query, dire cosa si intende per correttezza, completezza e freschezza di una computazione.
    \item Nell'ambito del problema della integrit`a del risultato di query, dire cosa si intende per completezza, correttezza
e freschezza del risultato di una query. Si richiede inoltre di descrivere una tecnica deterministica di controllo
dell'integrità.
\end{itemize}

\chapter{Domande pacchetto 3 - Query distribuite}
\begin{itemize}
    \item Nell'ambito di query distribuite, dire in che cosa consiste la tecnica per l'esecuzione di join detta sovereign join.
    \item Nell'ambito delle tecniche per l'esecuzione selettiva di query distribuite, descrivere cosa rappresenta il profilo
di una relazione nel caso in cui si utilizzi il modello con tre livelli di visibilità (no visibility, plaintext visibility,
encrypted visibility). Si richiede inoltre di mostrare un semplice esempio di query e autorizzazioni che richieda
l'uso della crittazione per poter essere eseguita.
    \item Nell'ambito delle tecniche per l'esecuzione selettiva di query distribuite, l'approccio basato sulla definizione di
viste si avvale di una riscrittura delle interrogazioni fatta in accordo al modello Truman oppure al modello
non-Truman. Si richiede di descrivere questi due modelli e, in particolare, di fare un esempio di riscrittura
basata sul modello Truman.
    \item Nell'ambito di query distribuite, quando due autorizzazioni definite come coppie [Attributi,Relazioni] possono
essere combinate in modo safe? Si richiede di fornire un esempio.
    \item Si richiede di descrivere il concetto di profilo di una relazione nell'ambito del modello di controllo dell'accesso
per query distribuite con uso di encryption. Inoltre si richiede di fare un esempio di interrogazione di join e di
mostrare il profilo delle relazioni di partenza e della relazione che si ottiene tramite l'esecuzione della query.
    \item Nell'ambito dei modelli di autorizzazione per query distribuite che supportano la specifica del join path, dire
cosa rappresenta il profilo di una relazione [$R_\pi$,$R_{\triangleright\triangleleft}$,$R_\sigma$] e fare un esempio di select-where query con associato
il profilo relativo al risultato della query.
\end{itemize}

\chapter{Domande pacchetto 4 - Differential privacy}
\begin{itemize}
    \item Dire formalmente quando un algoritmo (A) soddisfa la definizione di $\epsilon$-differential privacy.
    \item Nell'ambito del concetto di differential privacy dire cosa si intende per global sensitivity e fornire un esempio.
    \item Nell'ambito della differential privacy, cosa si intende per global sensitivity? Si richiede di fare un esempio e di
illustrare la relazione tra la global sensitivity ed il meccanismo di Laplace.
    \item Nell'ambito della differential privacy, si richiede di fornire la definizione formale di algoritmo che soddisfa la
definizione di $\epsilon$-differential privacy. Si richiede inoltre di descrivere, fornendo anche un esempio, cosa si intende
per global sensitivity e a cosa serve.
    \item Dire formalmente quando un algoritmo (A) soddisfa la definizione di $\epsilon$-differential privacy e spiegare la
definizione.
    \item Nell'ambito della differential privacy, cosa si intende per composizione sequenziale e composizione parallela? Fornire un esempio.
    \item Nell'ambito della differential privacy, cosa indica il parametro $\epsilon$?
    \item Nell'ambito della differential privacy, cosa si intende per modello interattivo e modello non interattivo?
    \item Quale è la differenza tra global differential privacy e local differential privacy?
    \item Cosa vuol dire che differential privacy è “chiusa rispetto a operazioni di post-processing”?
\end{itemize}


\chapter{Domande che penso siano di altri pacchetti}
\begin{itemize}
    \item Nell'ambito delle blockchain, dire in cosa consiste la tecnica detta proof-of-work. E' corretto affermare che
grazie a questa tecnica la scelta del nodo del sistema che crea un nuovo blocco è casuale?
    \item Nell'ambito delle tecniche per la specifica di preferenze utente per la selezione di cloud plan, descrivere i due
tipi di preferenze (su valori e su attributi) che possono essere espressi e fare un esempio.
    \item Nell'ambito delle blockchain, si richiede di descrivere (ad alto livello) come funziona il protocollo del consenso.
    \item Nell'ambito delle blockchain, si richiede di descrivere la struttura di un blocco.
    \item Nell'ambito delle tecniche per la specifica di preferenze utente per la selezione di cloud plan, descrivere i due
tipi di preferenze (su valori e su attributi) che possono essere espressi e fare un esempio.
    \item Nell'ambito delle blockchain, perchè è importante che ciascun blocco abbia nell'header l'hash del blocco
precedente? Mostrare un esempio della sua utilità.
    \item Nell'ambito della specifica degli approcci per esprimere preferenze da parte degli utenti su proprietà dei cloud
plan, si richiede di descrivere le tre techniche di ranking (pareto dominance, d-dominance, wd-dominance) e
fare degli esempi per ognuna di esse.
    \item Nell'ambito delle tecniche per la specifica di requisiti utenti per la scelta di cloud provider, perchè è importante
anche supportare, oltre che la specifica dei requisiti, anche la specifica di preferenze? Ad esempio, quali tipi di
preferenze possono essere specificate?
\end{itemize}


\noindent \textbf{Note:}
\begin{itemize}
    \item L'esame del 10/06/2024 è simile a quello che sarà il nostro
    \item L'esame del 3/07/2024 è simile a quello che sarà il nostro
\end{itemize}














\end{document}