\documentclass{report}
\usepackage{graphicx} % Required for inserting images
\usepackage[italian]{babel}
\usepackage{tikz}
\usepackage{hyperref}
\usepackage{amsmath}
\usepackage{xcolor}
\usepackage{float}
\usepackage{soul}
\usepackage{listings} % Per evidenziare il codice

\definecolor{lightgray}{rgb}{0.9,0.9,0.9} % Definizione colore sfondo
\definecolor{darkgreen}{rgb}{0.0, 0.5, 0.0}

\lstset{
    backgroundcolor=\color{lightgray}, % Sfondo grigio
    basicstyle=\ttfamily, % Font monospaziato
    % frame=single, % Bordo attorno al codice
    tabsize=4, % Dimensione tabulazione
    breaklines=true, % Permette di andare a capo automaticamente
    numbers = left,
    numberstyle=\small\color{gray}
}

\title{\huge\textbf{{Domande}}}
\date{Parte extra}
\begin{document}

\maketitle
\tableofcontents
\newpage

\chapter{Domande pacchetto 1}
\begin{itemize}
    \item Nell'ambito delle politiche basate sui ruoli, descrivere il principio di separazione dei privilegi (statico e dinamico)
e fornire un esempio per entrambe le tipologie.
    \item Nell'ambito delle politiche discrezionarie, dire cosa rappresentano le ACL e le capability. Illustrare i vantaggi
e gli svantaggi.
    \item Nell'ambito del modello relazionale multilivello, dire cosa si intende per tuple poliinstanziate ed elementi
poliinstanziati e fornire un esempio per entrambi i concetti.
    \item Descrivere, tramite un esempio, perchè le politiche discrezionarie sono vulnerabili a Trojan horse.
    \item Nell'ambito del controllo dell'accesso, definire il concetto di gruppo e ruolo, evidenziando quali sono le differenze
tra questi due concetti.
    \item Nell'ambito del modello di Biba, descrivere la politica low-water mark per oggetti. Questa politica garantisce
l'integrità delle informazioni? Si richiede di giustificare la risposta.
    \item 
\end{itemize}

\chapter{Domande pacchetto 2}
\begin{itemize}
    \item Nell'ambito delle tecniche usate per verificare l'integrità del risultato di query, dire quale è la differenza tra
tecniche deterministice e tecniche probabilistice. Si richiede inoltre di fornire un esempio di query e come si
può utilizzare una tecnica deterministica per verificarne il risultato della query di esempio.
    \item Nell'ambito delle tecniche per l'integrità del risultato di query, si richiede di descrivere l'approccio basato sul
Merkle-tree e di fornire un semplice esempio di verifica, specificando la query, la tabella su cui viene eseguita
eil Merkle tree costruito sulla tabell.
    \item 
\end{itemize}

\chapter{Domande pacchetto 3}
\begin{itemize}
    \item Nell'ambito di query distribuite, dire in che cosa consiste la tecnica per l'esecuzione di join detta sovereign join.
    \item Nell'ambito delle tecniche per l'esecuzione selettiva di query distribuite, descrivere cosa rappresenta il profilo
di una relazione nel caso in cui si utilizzi il modello con tre livelli di visibilità (no visibility, plaintext visibility,
encrypted visibility). Si richiede inoltre di mostrare un semplice esempio di query e autorizzazioni che richieda
l'uso della crittazione per poter essere eseguita.
    \item 
\end{itemize}

\chapter{Domande pacchetto 4}
\begin{itemize}
    \item Dire formalmente quando un algoritmo (A) soddisfa la definizione di $\epsilon$-differential privacy.
    \item Nell'ambito del concetto di differential privacy dire cosa si intende per global sensitivity e fornire un esempio.
    \item Nell'ambito della differential privacy, cosa si intende per global sensitivity? Si richiede di fare un esempio e di
illustrare la relazione tra la global sensitivity ed il meccanismo di Laplace.
    \item 
\end{itemize}


\chapter{Domande che penso siano di altri pacchetti}
\begin{itemize}
    \item Nell'ambito delle blockchain, dire in cosa consiste la tecnica detta proof-of-work. E' corretto affermare che
grazie a questa tecnica la scelta del nodo del sistema che crea un nuovo blocco è casuale?
    \item Nell'ambito delle tecniche per la specifica di preferenze utente per la selezione di cloud plan, descrivere i due
tipi di preferenze (su valori e su attributi) che possono essere espressi e fare un esempio.
    \item Nell'ambito delle blockchain, si richiede di descrivere (ad alto livello) come funziona il protocollo del consenso.
    \item Nell'ambito delle blockchain, si richiede di descrivere la struttura di un blocco.
    \item Nell'ambito delle tecniche per la specifica di preferenze utente per la selezione di cloud plan, descrivere i due
tipi di preferenze (su valori e su attributi) che possono essere espressi e fare un esempio.
    \item Nell'ambito delle blockchain, perchè è importante che ciascun blocco abbia nell'header l'hash del blocco
precedente? Mostrare un esempio della sua utilità.
\end{itemize}


\noindent prox: Esame 4/07/2023 














\end{document}