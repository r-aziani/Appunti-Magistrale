\chapter{Secret Sharing}

Con il termine \textit{secret sharing} si intendono i metodi per \textbf{distribuire 
un segreto tra un gruppo di partecipanti}, a ciascuno dei quali viene assegnato una 
parte di tale segreto; il segreto può essere ricomposto solo quando un numero 
sufficiente di parti viene combinato insieme.

\noindent In uno schema di condivisione segreta c'è un \textit{dealer} e \textit{n player}; il dealer 
fornisce una parte del segreto ai player, e solo quando determinate condizioni sono soddisfatte 
i player sono in grado di ricostruire il segreto.

\section{Schema (\textit{n,n})}

Sono richiesti $n$ player su $n$ partecipanti per ricostruire il segreto (quindi tutti);
il processo di distribuzione funziona le modo seguente:
\begin{itemize}
    \item viene scelto un numero primo $p$
    \item si sceglie un numero segreto $s$ con $s<p$
    \item si scelgono casualmente $n-1$ valori minori di $p$, ciascuno dei quale sarà uno 
    share $a_1, a_2, \dots, a_{n-1}$
    \item lo share $a_n$ viene calcolato come $a_n = (s- a_1 - a_2 - \dots - a_{n-1})$ $mod$ $p$
    \item viene distribuito a ciascuno utente uno share $a_i$; segue che per ricostuire $s$ servono 
    tutti gli share 
\end{itemize}

\section{Schema (\textit{k,n})}

In questo caso il segreto viene diviso in $n$ parti, ma ne bastano $k$ per ricostruirlo; il 
funzionamento è il seguente:
\begin{itemize}
    \item bisogna scegliere un primo $p$ e un segreto $s$ tale che $s<p$
    \item si generano $k-1$ valori $a_1, \dots, a_{k-1}$ che compongono lo share
    \item l'\textit{i}-esimo share si ottiene come:
    \begin{center}
        $y_i = (s + a_1 \cdot i + a_2 \cdot i^2 + \dots + a_{k-1} \cdot i^{k-1})$ $mod$ $p$
    \end{center}

    con $i$ che assume valori da 1 a $n$, corrisponde al numero dello share
\end{itemize}

\noindent La ricostruzione del segreto avviene risolvendo un sistema di $k$ equazioni in $k$
incognite ($s$ e $k-1$ valori di $a$).

