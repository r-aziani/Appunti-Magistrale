\chapter{Certificati}

Un certificato è un documento elettronico che attesta \textbf{l'associazione univoca tra 
una chiave pubblica e l'identità di un soggetto} (persona, compagnia, computer \dots), il quale 
dichiara di usarla in procedura di cifratura asimmetrica e/o autenticazione tramite 
firma digitale.

\noindent Il certificato contiene informazioni sulla chiave, sull'identità del proprietario 
e la firma digitale di un'autorità emittente che ha verificato il contenuto del certificato. I 
certificati si rivelano utili quando la chiave pubblica viene usata su larga scala.

\noindent Proprietà del certificato:
\begin{itemize}
    \item chiunque deve poter leggere e determinare nome e chiave pubblica 
    \item chiunque deve poter verificare l'autenticità
    \item solo l'autorità di certificazione (CA) può creare e aggiornare i certificati 
\end{itemize}


\noindent Un certificato può contenere altre informazioni, come ad esempio:
\begin{itemize}
    \item periodo di validità
    \item id della chiave 
    \item stato della chiave 
    \item informazioni addizionale su chiave e utente
\end{itemize}

\noindent Dato che i certificati possono essere revocati, si deve verificare questa evenienza 
consultando una lista dei certificati revocati (CRL), che deve essere aggiornata dalla CA. Una 
revoca può essere richiesta per diversi motivi:
\begin{itemize}
    \item compromissione della chiave privata 
    \item informazioni non più valide 
    \item compromissione dell'algoritmo usato 
\end{itemize}


'