\chapter{Introduzione}

La crittografia
fa parte, insieme alla crittoanalisi, della crittologia.

\noindent La \textbf{crittografia} è la scienza che studia tecniche e metodologie per \textbf{cifrare un testo in chiaro}, al fine di produrre un 
\textbf{testo cifrato comprensibile solo ad un ricevente legittimo}, il
quale possiede l’informazione sufficiente (detta chiave) per decifrarlo, recuperando il testo in chiaro. 

\noindent La \textbf{crittoanalisi} studia come decrittare un testo cifrato per ottenere il testo in chiaro: il verbo decrittare indica l’azione
compiuta da un’entità che non possiede la chiave per recuperare, in modo legittimo, il testo in chiaro. In letteratura,
suddetta entità viene indicata con il termine ascoltatore, oppure avversario o anche nemico. Lo scopo
della crittografia è di produrre un messaggio cifrato $m$ in modo che nessun avversario sia in grado di decrittarne il contenuto.

\noindent La \textbf{stegonografia} (dal greco \textit{scrivere nascosto}), indica l'insieme delle tecniche che permettono a due (o più) 
entità in modo tale da occultare, agli occhi di un ascoltatore, non tanto il contenuto (come nel caso della crittografia), ma l'estistenza 
stessa di una comunicazione. In altre parole, è l'arte di nascondere un messaggio all'interno di un altro \textit{messaggio contenitore}. Alcuni esempi sono:
\begin{itemize}
    \item disposizione dei caratteri 
    \item inchiostro invisibile
    \item contrassegno dei caratteri 
    \item nascondere messaggi all'interno di bit di file multimediali 
    \item \dots
\end{itemize}

\subsubsection{Proprietà di sicurezza di un messaggio}

\begin{enumerate}
    \item \textbf{Confidenzialità}
    \item \textbf{Autenticazione}
    \item \textbf{Non ripudio}
    \item \textbf{Integrità}
    \item \textbf{Anonimia} (canali in cui le entità comunicanti devono poter nascondere la loro identità)
\end{enumerate}

