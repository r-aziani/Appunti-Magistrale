\chapter{Composizione di modelli: ASM multi agenti}
\noindent Grazie alle regole per la strutturazione, è possibile definire un sistema distribuito
come una composizione di ASM.
\begin{itemize}
    \item Ogni componente del sistema distribuito viene
    modellato mediante una opportuna ASM
    \item Risulta un sistema di ASM Distribuite, dette anche \textbf{ASM multi agenti}
\end{itemize}

\section{ASM multi agenti}
\noindent Descrivono un modello distribuito della computazione. La computazione distribuita è modellata
mediante un insieme di Agenti che operano in modo concorrente
\begin{itemize}
    \item \textit{Movimenti} concorrenti sincroni/asincroni
    \item \textit{Stati globali} condivisi tra gli agenti
\end{itemize}

\noindent Gli agenti ASM:
\begin{itemize}
    \item Possono essere creati dinamicamente
    \item Hanno una visione parziale dello stato globale
    \item Hanno il proprio programma da eseguire
\end{itemize}

\noindent Le ASM multi agenti sono classificate in:
\begin{itemize}
    \item \textbf{Sincrone:} gli agenti eseguono il loro programma in parallelo, sincronizzati su un implicito clock globale del sistema
    \item \textbf{Asincrone:} gli agenti eseguono il loro programma in parallelo, ma in modo indipendente tra loro 
    \begin{itemize}
        \item Ciascuno ha il \textit{proprio clock} che regola la durata du una mossa
        \item Ciascuno opera nel \textit{proprio stato locale}
    \end{itemize}
\end{itemize}

\subsection{Definizione ASM multiagente:}
\noindent 
\begin{center}
    \textit{Una sync/async ASM è un insieme di coppie (a, ASM(a)) dove
    \begin{itemize}
        \item di agenti a $\in$ Agent (il dominio degli agenti)
        \item e programmi ASM(a) che sono ASM di base
    \end{itemize}}
\end{center}

\subsection{Codifica AsmetaL di ASM Multiagenti}
\begin{center}
    $[asyncr] asm ASM-name$
\end{center}
\noindent a parola chiave asyncr è opzionale(perchè racchiusa nelle parentesi quadre)
\begin{itemize}
    \item Se presente, denota una ASM \textit{asincrona} multi-agent
    \item Se omessa, l’ASM è considerata \textit{sincrona} multi-agent
\end{itemize}

\begin{itemize}
    \item Ogni agente ha una \textit{visione parziale} dello stato globale
    \begin{itemize}
        \item \textit{View(a, M)} indica la vista dell’agente a dello
        stato globale di una ASM M
    \end{itemize}
    \item Ogni agente può avere una \textit{visione privata} non condivisa con altri agenti
    \begin{itemize}
        \item per una $f : A-> B$ globale, la dichiarazione privata di f diventa $ f : Agent$ x $A->B$ 
        \begin{itemize}
            \item f(self,x) denota la \textit{verisone privata di f(x)} dell'agente corrente \textit{self}
        \end{itemize}
        \item \textit{self} viene interpretata da \textit{a} (agente) come se stesso
    \end{itemize}
\end{itemize}

