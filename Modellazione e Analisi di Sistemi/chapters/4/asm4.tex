\chapter{Specifica di protocolli di sicurezza: il protocollo di Needham Schroeder}
\section{Crittografia: concetti base}
\begin{itemize}
    \item \textbf{Algoritmi crittografici:} algoritmi che trasformano un testo (o una info) in un testo cifrato utilizzando chiavi
    \item \textbf{Protocolli crittografici:} sequenze di scambio di messaggi a cui gli agenti possono applicare
    algoritmi crittografici
    \begin{itemize}
        \item \textbf{a chiave simmetrica:} la stessa K è utilizzata per criptare e decriptare
        \item \textbf{a chiave pubblica:} K-pub per criptare e K-priv per decriptare
        \item \textbf{a chiave condivisa:} una K-AB tra 2 agenti fornita da un terzo
    \end{itemize}
\end{itemize}

\subsubsection{Protocolli per l'autenticazione}
\noindent Lo scopo è \textbf{l'autenticazione}
\begin{itemize}
    \item fornire ad una parte la certezza dell'identità del mittente del messaggio ricevuto 
    \item eventualmente stabilire una chiave (o altra info) comune 
\end{itemize}

\noindent \textbf{Segretezza:} in più possono garantire che l'informazione (o parte) scambiata è rimasta segreta

\subsubsection{Assunzioni di base}
\begin{itemize}
    \item \textbf{Forza della spia:} la spia (man in the middle) può
    \begin{itemize}
        \item intercettare (ed eventualmente analizzare) ogni messaggio che sia stato spedito da un agente
        \item formare (sintetizzare) nuovi messaggi utilizzando la sua base di conoscenze
        \item cambiare destinatario di un messaggio
    \end{itemize}
    \item \textbf{Forza della crittografia:} 
    \begin{itemize}
        \item un messaggio può essere decriptato solo se si ha la chiave giusta
        \item un agente può generare sempre nonces nuove
        \item un agente autenticato è non compromesso se la sua chiave privata non è nota alla spia
    \end{itemize}
\end{itemize}

\section{Needham-Schroeder a chiavi pubbliche}