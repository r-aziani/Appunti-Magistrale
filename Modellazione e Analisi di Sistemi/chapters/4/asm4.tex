\chapter{Specifica di protocolli di sicurezza: il protocollo di Needham Schroeder}
\section{Crittografia: concetti base}
\begin{itemize}
    \item \textbf{Algoritmi crittografici:} algoritmi che trasformano un testo (o una info) in un testo cifrato utilizzando chiavi
    \item \textbf{Protocolli crittografici:} sequenze di scambio di messaggi a cui gli agenti possono applicare
    algoritmi crittografici
    \begin{itemize}
        \item \textbf{a chiave simmetrica:} la stessa K è utilizzata per criptare e decriptare
        \item \textbf{a chiave pubblica:} K-pub per criptare e K-priv per decriptare
        \item \textbf{a chiave condivisa:} una K-AB tra 2 agenti fornita da un terzo
    \end{itemize}
\end{itemize}

\subsubsection{Protocolli per l'autenticazione}
\noindent Lo scopo è \textbf{l'autenticazione}
\begin{itemize}
    \item fornire ad una parte la certezza dell'identità del mittente del messaggio ricevuto 
    \item eventualmente stabilire una chiave (o altra info) comune 
\end{itemize}

\noindent \textbf{Segretezza:} in più possono garantire che l'informazione (o parte) scambiata è rimasta segreta

\subsubsection{Assunzioni di base}
\begin{itemize}
    \item \textbf{Forza della spia:} la spia (man in the middle) può
    \begin{itemize}
        \item intercettare (ed eventualmente analizzare) ogni messaggio che sia stato spedito da un agente
        \item formare (sintetizzare) nuovi messaggi utilizzando la sua base di conoscenze
        \item cambiare destinatario di un messaggio
    \end{itemize}
    \item \textbf{Forza della crittografia:} 
    \begin{itemize}
        \item un messaggio può essere decriptato solo se si ha la chiave giusta
        \item un agente può generare sempre nonces nuove
        \item un agente autenticato è non compromesso se la sua chiave privata non è nota alla spia
    \end{itemize}
\end{itemize}

\section{Needham-Schroeder a chiavi pubbliche}
\noindent \textbf{Obiettivo:} stabilire mutua autenticazione tra due agenti A e B (e una nonce segreta comune)
\subsubsection{Protocollo}
\begin{enumerate}
    \item \textbf{A} che vuole iniziare la comunicazione con \textbf{B}, manda una nuova nonce a \textbf{B}:
    \begin{center}
        $A->B :$ \{Na, A\}Kb
    \end{center}
    \item \textbf{B} legge il messaggio, rimanda la nonce di \textbf{A} più un'altra nuova:
    \begin{center}
        $B->A :$ \{Na, Nb\}Ka
    \end{center}
    \item \textbf{A} legge il messaggio, riconosce la nonce rispedita da \textbf{B} (solo B può averla letta) e rimanda a \textbf{B}, Nb:
    \begin{center}
        $A->B :$ \{Nb\}Kb
    \end{center}
    \item \textbf{B} ricevendo indietro la nonce Nb sa che l'autenticazione (handshake) è avvenuta (solo \textbf{A} può averla letta)
\end{enumerate}

\subsubsection{Vulnerabilità}
\noindent Il protocollo è suscettibile ad attacchi di tipo \textit{man in the middle}:
\begin{itemize}
    \item un impostore Spy riesce ad avere il controllo del traffico (modellato con Alice che inizia una sessione di comunicazione con Spy)
    \item inoltra i messaggi a Bob convincendolo di essere in comunicazione con Alice ed inganna Alice facendole credere di comunicare con Bob     
\end{itemize}

\noindent Spy agisce da "uomo nel mezzo" e intercetta tutti i messaggi di Bob, che crede di essere in
comunicazione sicura con Alice

\noindent L'attacco è stato descritto per la prima volta da Gavin Lowe nel 1995

\subsection{Attacco di Lowe}
\noindent \textbf{A} inizia una istanza del protocollo con la spia (o con un agente compromesso)
\begin{center}
    $A-> Spy :$ \{Na, A\}Ks
    
    $Spy->B :$ \{Na, A\}Kb
    
    $B->A :$ \{Na, Nb\}Ks
    
    $Spy->B :$ \{Nb\}Kb
\end{center}

\noindent La spia acquisisce una informazione segreta di B: \textbf{Nb è compromessa}

\noindent A sta usando una nonce compromessa

\noindent B ha una nonce Nb compromessa, eppure né A né B sono compromessi

\noindent Per garantire l'autenticazione il protocollo richiederebbe
\begin{itemize}
    \item che gli agenti non siano compromessi ma
    \item anche che non abbiano iniziato comunicazioni
    con agenti compromessi
\end{itemize}

\subsubsection{La soluzione di Lowe}
\begin{enumerate}
    \item \textbf{A} che vuole iniziare la comunicazione con \textbf{B}, manda una nuova nonce a \textbf{B}:
    \begin{center}
        $A->B :$ \{Na, A\}Kb
    \end{center}
    \item \textbf{B} legge il messaggio, rimanda la nonce di \textbf{A} più un'altra nuova e \underline{\textbf{la sua identità}}:
    \begin{center}
        $B->A :$ \{Na, Nb, B\}Ka
    \end{center}
    \item \textbf{A} legge il messaggio, riconosce la nonce rispedita da \textbf{B} (solo B può averla letta) e rimanda a \textbf{B}, Nb:
    \begin{center}
        $A->B :$ \{Nb\}Kb
    \end{center}
    \item \textbf{B} ricevendo indietro la nonce Nb sa che l'autenticazione è avvenuta (solo \textbf{A} può averla letta)
\end{enumerate}

\subsubsection{Esercizio NS in ASM}